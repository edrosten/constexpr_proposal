\documentclass[prd,twocolumn,amsmath,amssymb,nofootinbib,eqsecnum]{revtex4-1}
\it
\usepackage{xspace} %Dude this will blow your mind!
\usepackage{amsthm}
\usepackage{color}
\usepackage{soul}
\usepackage{fancyvrb}
\usepackage{epsfig}
\usepackage{hyperref}

\newcommand{\constexpr}{\code{constexpr}\xspace}
\newcommand{\code}[1]{{\tt #1}}
\newcommand{\header}[1]{{\tt <#1>}}
\newcommand{\cmath}{\header{cmath}}
\newcommand{\complex}{\header{complex}}
\newcommand{\cstdlib}{\header{cstdlib}}

\newcommand{\ie}{i.e.}

\newcommand{\FEINVALID}{{\tt FE\_INVALID}}
\newcommand{\FEDIVBYZERO}{{\tt FE\_DIVBYZERO}}
\newcommand{\FEINEXACT}{{\tt FE\_INEXACT}}
\newcommand{\FEUNDERFLOW}{{\tt FE\_UNDERFLOW}}
\newcommand{\FEOVERFLOW}{{\tt FE\_OVERFLOW}}
\newcommand{\FLTRADIX}{{\tt FLT\_RADIX}}
\newcommand{\Operators}{\ensuremath{+,-,\times,/}}

\newcommand{\highlight}[1]{\colorbox{green}{\!\!\!\! #1}}
\newcommand{\oldhighlight}[1]{\colorbox{blue}{\!\!\!\! #1}}
\newcommand{\stdcomment}[1]{{// {\it see} [#1]}}

\newcommand{\eq}[1]{(\ref{eq:#1})}
\newcommand{\fig}[1]{figure~\ref{fig:#1}}
\newcommand{\sect}[1]{section~\ref{sec:#1}}

\newcommand{\bigund}{{\Huge{\_}}}

\newtheorem*{proposal*}{Proposal}

\makeatletter
\def\l@subsection#1#2{}
\def\l@subsubsection#1#2{}
\makeatother

\def\bibsection{\section*{\refname}} 

\begin{document}


\title{More \constexpr for \cmath\ and \complex}
%\author{Edward J.~Rosten \& Oliver J.~Rosten}
%\date{\today}

\author{
\hspace{11.5em}
\begin{tabular}{ll}
	Document: & D1383R1
\\
	Date: & \today
\\
	Project: & Programming Language C++, Library Working Group
\\
	Audience: & SG6 $\rightarrow$ LEWG $\rightarrow$ LWG $\rightarrow$ CWG
\\
	Reply to: & Edward J.\ Rosten {(\tt erosten@snap.com)}
 / Oliver J.\ Rosten {(\tt oliver.rosten@gmail.com)}
\end{tabular}
}

\begin{abstract}

\begin{center} {\bf Abstract} \end{center}

A scattering of $\constexpr$, principally throughout \cmath, was proposed in~\cite{Rosten-constexpr} and accepted into C++23. This was subject to a constraint that the affected functions be limited to those which are, in a well-defined sense, no more complicated than the arithmetic operators \Operators. It is proposed to remove this restriction, thereby allowing a richer spectrum of mathematical functions to be used in a \constexpr\ context. 

%The focus of this proposal is to build upon existing practice, with the goal of providing something immediately useful to a section of the C++ user base. However, it is recognized that the mathematical functions considered do not serve those requiring strict platform-independence. It is recognized that this is an important need, but argued that it is orthogonal and therefore befits a separate proposal.

\end{abstract}


\maketitle
\tableofcontents

\section{Introduction}
\label{sec:Intro}

Since its inception, \constexpr\ has become an invaluable ingredient in compile-time programming. Indeed, part of its appeal is that the sharp distinction between meta-programming and runtime programming has in many instances become blurred. The interest in \constexpr\ is reflected by the numerous papers proposing to increase the range of core language features and library functionality that may be used in a \constexpr\ context. As such, it is essential for the long-term uniformity of C++ that parts of the standard library are not left behind in this process.

This paper is the natural extension of~\cite{Rosten-constexpr} and seeks to significantly expand the number of functions in \cmath\ (and also \complex) which may be used in a \constexpr context. The potential utility of this for numerics is noteworthy. However, it is clear that there are new hurdles to overcome: floating-point is very subtle and different people may want implementations to prioritize different things~\cite{Timmons-Less}. In other words, there is a non-trivial design space. 

Consider some function $f(x)$ and its floating-point implementation $F_R(x_n)$, where the $x_n$ are numbers representable by the floating-point type being used and $R$ denotes the rounding mode. A central issue is whether, for a given $x_n$ and $R$, $F_R(x_n)$ should always give the same answer, regardless of compiler settings and/or platform. In some cases, the answer is clearly yes: for example if $f$ corresponds to multiplication by two.  However, to take a different extreme, what about \code{std::sin(1e100)}?%
\footnote{Thanks to Richard Smith for first bringing this example to our attention.}

Whilst it is of course true that there is an unambiguous result to $\sin 10^{100}$, we may well wonder how meaningful it is, since shifting the argument by a tiny amount---of relative size $10^{-100}$---can cause the output to change dramatically. The good news is that this intuition can be quantified. Floating-point numbers can be thought of in two ways: as points on the number line or as intervals (c.f.\ interval arithmetic~\cite{Interval}). For a given $x_n$ and rounding mode, the corresponding interval is such that all numbers in the interval collapse to $x_n$. The granularity of floating-point is such that $\sin x$ sweeps across its entire range from [-1,+1] many times in the interval containing $10^{100}$. Contrariwise, multiplication by two maps the interval associated with $x_n$ into the interval associated with $2x_n$.

Thus, within the interval approach, it is reasonable for different implementations to produce different
values for the same input; or even for the same implementation to do so depending on things such as the degree of optimization. This is true of the the way C++ works at present.

However, even within the interval approach, it is not a free-for-all. Under the action of the true mathematical function, $f$, the floating-point interval containing $x$ is mapped into some union of floating-point intervals. The latter define what can reasonably be considered equally valid output of $F_R(x_n)$. Does C++ conform to this? Surely not! Attempting to remedy this via standardization seems
counterproductive. The effect would likely be to render all the existing implementations---which for many purposes work perfectly well---non-conforming. Rather, it is suggested that consideration of intervals can
be optionally used to quantify the Quality of Implementation.

With all this in mind, the design space divides up as follows:
\begin{enumerate}
	\item[Canonical] The output of mathematical functions is exactly prescribed, down to the last bit.
	
	\item[Interval] The output of mathematical functions may fall within a range, generated by considering the mapping of intervals corresponding to floating-point numbers.
	
	\item[Approximate] A weaker version of [Interval], where deviations from the desired output range 
	are taken to reflect the QoI.
\end{enumerate}

C++ currently falls within the last of these. Is this a problem? It depends. If you are writing a networked multiplayer game, then lack of support for [Canonical] may be problematic. On the other hand, if you are doing physical modelling where the errors in the boundary conditions are large compared to the floating-point granularity at the appropriate scale then [Approximate] may be perfectly sound. Indeed, [Approximate] may allow vendors to write faster implementations than [Canonical] and why should users pay for a spurious 'improvement' in accuracy?

All of this suggests (at least) two avenues for standardization.
\begin{enumerate}
	\item A new set of mathematical functions, satisfying the requirements of [Canonical].
	
	\item Allowing existing [Approximate] implementations to be used in a \constexpr\ context.
\end{enumerate}

This paper proposes standardization of the latter. To conclude the introduction, given our belief that declaring more functions within \cmath\ to be \constexpr\ is useful, we seek to adhere to one of the core principles of C++~\cite{DandE}:
\begin{quote}
	It is more important to allow a useful feature than to prevent every misuse.
\end{quote}

\section{Motivation \& Scope}
\label{sec:Motivation}

Prior to~\cite{Rosten-constexpr}, no effort had been made to allow for functions in \cmath\ to be declared \constexpr; this despite there being glaring instances, such as \code{std::abs}, for which this was arguably perverse. Indeed, between~\cite{AP-complex} and~\cite{Rosten-constexpr} being adopted, the situation was actually been better for \complex\ than for \cmath! The aim of~\cite{Rosten-constexpr} was to at least partially rectify the situation, while recognizing that attempting to completely resolve this issue in a single shot was too ambitious.

The broad strategy of~\cite{Rosten-constexpr} is to focus on those functions which are, in a well-defined sense, no more complicated than the arithmetic operators \Operators; the rationale for this being that the latter are already available in a \constexpr\ context.
As~\cite{Rosten-constexpr} proceeded through the standardisation process, LEWG expressed a desire 
to extend the scope to include a significant amount of what remains in \cmath, in particular common mathematical functions such as \code{std::exp}.%
\footnote{It seems too ambitious at this stage to include the mathematical special functions [sf.cmath] and so they are excluded from this proposal.}
However, later discussion---crystalized in~\cite{Timmons-Less}---revealed significant concerns. Our hope is that these concerns can be overcome; to aid meaningful discussion, this proposal devotes space to describing a framework
for rigorously conceptualizing the various issues.

There are several related concerns to the goal of this paper to declare more functions in \cmath\ to be \constexpr:
\begin{enumerate}
	\item Implementations of certain functions in \cmath\ do not produce results which are `correctly' rounded to the last bit.
	
	\item The output of certain functions in \cmath\ may depend on the level of optimization; a corollary
	is that even with a given level of (non-trivial) optimization, the same function in \cmath\ may give 
	different answers, depending on the ambient \href{https://godbolt.org/z/js7rGvPbf}{code}.%
	\footnote{Thanks to Matthias Kretz for supplying this example.}
	
	\item The output of certain functions in \cmath\ may differ when the same binary is executed on
	different CPUs within the same architectural family.
		
	\item The output of certain functions in \cmath\ may differ depending on whether they are evaluated at runtime or translation time.
\end{enumerate}

A conceptual framework for reasoning about these issues has been given in the introduction. If one had the luxury of possessing both a [Canonical] and [Interval] implementation then much of the concern could be allayed by simply choosing the appropriate implementation for the task in hand: in the case of [Interval] the same function may produce different results in different contexts, but it a manner such that all outputs are, in a well defined sense, equally valid.

However, the reality is that the only available implementations are [Approximate]. Furthermore, even if an [Interval] implementation were available, some disquiet may follow from the fact that different platforms
may do different things in the following example:
\begin{verbatim}
template<double D>
struct do_stuff
{
  static void execute() {}
};

template<>
struct do_stuff<1.0>
{
  static void execute() { destroy_everything(); }
};

// Do I feel lucky?
do_stuff<std::sin(1e100)>::execute();
\end{verbatim}

However, we believe that the utility of rolling out \constexpr\ to touch more of \cmath\ outweighs the fact that it may be misused, bringing us back to the core principle cited in the introduction, from~\cite{DandE}.
As for whether certain results may come as a surprise to users, again this is true. But that
is true of floating-point, period. It is a difficult, subtle area of programming fraught with difficulty
due to its frequently counter-intuitive nature.

Is it perfectly reasonable for people to want to generate a \constexpr\ lookup table for (say) \code{std::sin}. If they are operating in the domain where they do not care whether their values differ between platforms etc.\ is it really right that we continue to prohibit this entirely legitimate use-case, for fear of abuse of floating-point?

\section{State of the Art and Impact on Implementers}

\subsection{Current Implementations}

With the exception of the special functions [sf.cmath], functions taking a pointer argument and those with an explicit dependence on the runtime rounding mode, GCC currently renders almost everything in \cmath\ \constexpr. Though clang does not have \constexpr implementations, it does perform compile time evaluation of many mathematical functions (but not the special functions) during optimization. The existence of compile time evaluation in GCC and clang demonstrates that implementation of this proposal is plausibly feasible. 

Nevertheless, even for GCC's implementation of the relatively simple functions which~\cite{Rosten-constexpr} declares \constexpr, there are subtleties. In particular GCC is not entirely consistent with the way in which it presently deals with NaNs and/or infinities when they are passed as arguments to various mathematical functions.


\subsection{Special Values}

Two problems that~\cite{Rosten-constexpr} had to deal with was situations in which
\begin{enumerate}

\item Floating-point exceptions (other than \FEINEXACT) are raised;

\item NaNs and/or infinities are passed as arguments to functions in \cmath\ declared \constexpr.

\end{enumerate}
The chosen solution was to delegate to Annex F of the C standard insofar as it applicable.
Recall that Annex F specifies C language support for IEC 60559 arithmetic; thus, to the extent that a
floating-point type conforms to this, the behaviour in the aforementioned situations is 
exactly prescribed in C++, following the adoption of~\cite{Rosten-constexpr}. Should a
floating-point type not conform to relevant parts of IEC 60559, then its behaviour in these
situations is unspecified.

This strategy is applicable in its current form to this paper, though the range of scenarios
in which Annex F may be invoked is somewhat richer. For example, an implementation
conforming to IEC 60559 must give \code{acos(1) = +0}.

\subsection{Interaction with the C Standard Library}

For a mathematical function which may be evaluated at translation time, 
putting all peculiarities of floating-point momentarily to one side, it is desirable for there to be
consistency with the values computed at runtime. However, the fact that the rounding mode
may be changed at runtime indicates that this is not, in general, possible. 

However, for more complicated mathematical functions there is an additional subtlety due
to the interaction with the C standard library. In [library.c] it is noted that \cmath\ makes available the facilities of the C standard library. One interpretation of this is that the C++
implementation could use one of several different C standard libraries. If so, 
constraining translation time behaviour so that it is consistent with the runtime behaviour
could be very difficult, quite apart from the issue of the runtime rounding mode.

Let us return to an earlier example:
\begin{Verbatim}
#include <cmath>
double f() { return std::sin(1e100); }
\end{Verbatim}
It turns out that on clang (targeting x64), the following code is emitted:
\begin{Verbatim}
.LCPI0_0:
	.quad	-4622843457162800295 
_Z1fv:
	movsd	.LCPI0_0(%rip), %xmm0
	retq
\end{Verbatim}
with equivalent code generated by GCC. This demonstrates that both compilers are already generating the results at translation time and, therefore, independently of the runtime C library. For this particular example, it appears that current practise does indeed achieve consistency between translation time and runtime, though effectively by ignoring the latter!

The story does not end here. For more complicated examples and/or removing optimization, it may be that a runtime call to the C library is made, after all. Bearing in mind that any value in the range [-1, 1] could be considered reasonable, this implies that the value of, say, \code{std::sin(1e100)} evaluated in one part of a code base may be very different from the (translation time) value evaluated elsewhere. 
Nevertheless, it seems reasonable in our opinion that both clang and GCC tacitly allow this, as already discussed in detail.



\section{Design Decisions}

The key design decision advocated in this paper is that it is acceptable for evaluation of a mathematical function to differ between translation time and runtime. Let us recapitulate the various points.

\begin{enumerate}
	\item Allowing a broader range of mathematical functions to be used within constant expressions is useful.
	
	\item Since the advent of \constexpr, the standard has implicitly allowed for differences between translation time and runtime evaluation: the arithmetic operators \Operators\ may be used in either context, but only in a runtime context may the rounding mode be changed.
	
	\item Even without \constexpr, current practice has long allowed for difference in the output of mathematical functions between any of translation time, runtime, runtime with different compiler flags, and runtime on a different platform. For example, optimization may emit code which entirely bypasses runtime calls to the C library, instead generating results at translation time. However, under other circumstances, optimization might not do this.
	
	\item The philosophy of this paper is not to accept an impasse. Rather, it is preferred to support
	a useful extension to existing practice in a non-prejudicial fashion, while not precluding 
	orthogonal developments which may cater for a different range of use-cases.
\end{enumerate}



One way or another, all of this boils down to the question of whether it is really acceptable for mathematical functions to give different results in different contexts, given the same input. Again, our answer is yes and we emphasise again that this is already part of C++. That being said, there are cases where people may want a mathematical function to produce the same result, given the same input, in all situations (except, presumably, when the rounding mode is changed). Our opinion is that this is best served by a separate proposal, which
perhaps introduces new types. It is worth pointing out that Java gives a good idea of what this
may look like.


\section{Impact On the Standard}

This proposal amounts to a (further) liberal sprinkling of \constexpr\ in \cmath, together with a smattering in \complex.

\section{Future Directions}

It would be desirable for library vendors to quantify the QoI of mathematical functions, along the lines indicated in \sect{Intro}. Orthogonal to this, it is worth considering separate implementations of mathematical functions with strict guarantees on their outputs. Finally, it would be ultimately desirable to extend \constexpr\ to some, if not all, of the special functions. 


\begin{acknowledgments}
	We would like to thank Richard Smith for his usual perceptive comments and Matthias Kretz and 	Nick Timmons for some very helpful feedback and discussions.
\end{acknowledgments}


\begin{thebibliography}{1}
	\bibitem[P0533]{Rosten-constexpr} Edward J.~Rosten and Oliver J.~Rosten, \constexpr\ for \cmath\ and \cstdlib.

	\bibitem[P2337]{Timmons-Less} Nicholas G. Timmons, Less \constexpr\ for \cmath.
	
	\bibitem[D\&E]{DandE} Bjarne Stroustrup, The Design and Evolution of C++.

	\bibitem[P0415R0]{AP-complex} Antony Polukhin, Constexpr for std::complex.	
		
	\bibitem[N4910]{WorkingPaper} Thomas K\"oppe, ed., Working Draft, Standard for Programming Language C++.	
	
	\bibitem[Kahan]{Interval} William Kahan. 2006. How Futile Are Mindless Assessments of Roundoff in Floating-Point Computation? Retrieved \today\ from \url{https://people.eecs.berkeley.edu/?wkahan/Mindless.pdf}.
	
	%\bibitem[Rudin 1976]{BabyRudin} Walter Rudin, Principles of Mathematical Analysis, 3rd Edition.
\end{thebibliography}

\newpage

%\begin{widetext}
\onecolumngrid

\section{Proposed Wording}

\setlength{\parindent}{0pt}


The following proposed changes refer to the Working Paper~\cite{WorkingPaper}. Highlighting in \highlight{green} indicates changes proposed in this paper, whilst \oldhighlight{blue} indicates changes proposed in the companion paper, \cite{Rosten-constexpr}.



\subsection{Modifications to ``Header \header{complex} synposis'' [complex.syn]}

// [complex.value.ops], {\it values }

\vspace{2ex}

\code{
  	template<class T> constexpr T real(const complex<T>\&);
	
  	template<class T> constexpr T imag(const complex<T>\&);

	\vspace{2ex}

	template<class T> \highlight{constexpr} T abs(const complex<T>\&);
 	
	template<class T> \highlight{constexpr} T arg(const complex<T>\&);
	
	template<class T> constexpr T norm(const complex<T>\&);
	
	\vspace{2ex}
	
	template<class T> constexpr conj(const complex<T>\&);
	
	template<class T> \highlight{constexpr} proj(const complex<T>\&);
	
	template<class T> \highlight{constexpr} polar(const T\&, const T\& = T());	

}

\vspace{2ex}

// [complex.transcendentals], {\it transcendentals}

\code{

\vspace{2ex}
  template<class T> \highlight{constexpr} complex<T> acos(const complex<T>\&);

  template<class T> \highlight{constexpr} complex<T> asin(const complex<T>\&);

  template<class T> \highlight{constexpr} complex<T> atan(const complex<T>\&);

\vspace{2ex}

  template<class T> \highlight{constexpr} complex<T> acosh(const complex<T>\&);

  template<class T> \highlight{constexpr} complex<T> asinh(const complex<T>\&);

  template<class T> \highlight{constexpr} complex<T> atanh(const complex<T>\&);

\vspace{2ex}

  template<class T> \highlight{constexpr} complex<T> cos  (const complex<T>\&);

  template<class T> \highlight{constexpr} complex<T> cosh (const complex<T>\&);

  template<class T> \highlight{constexpr} complex<T> exp  (const complex<T>\&);

  template<class T> \highlight{constexpr} complex<T> log  (const complex<T>\&);

  template<class T> \highlight{constexpr} complex<T> log10(const complex<T>\&);

\vspace{2ex}

  template<class T> \highlight{constexpr} complex<T> pow  (const complex<T>\&, const T\&);

  template<class T> \highlight{constexpr} complex<T> pow  (const complex<T>\&, const complex<T>\&);

  template<class T> \highlight{constexpr} complex<T> pow  (const T\&, const complex<T>\&);

\vspace{2ex}

  template<class T> \highlight{constexpr} complex<T> sin  (const complex<T>\&);

  template<class T> \highlight{constexpr} complex<T> sinh (const complex<T>\&);

  template<class T> \highlight{constexpr} complex<T> sqrt (const complex<T>\&);

  template<class T> \highlight{constexpr} complex<T> tan  (const complex<T>\&);

  template<class T> \highlight{constexpr} complex<T> tanh (const complex<T>\&);
}

\subsection{Modifications to  ``Header \header{cmath} synopsis'' [cmath.syn]}


\code{

namespace std\{

\vspace{2ex}
\ldots
\vspace{2ex}

%% acos %%

\highlight{constexpr} float acos(float x); \stdcomment{library.c}

\highlight{constexpr} double acos(double x);

\highlight{constexpr} long double acos(long double x); \stdcomment{library.c}

\highlight{constexpr} float acosf(float x);

\highlight{constexpr} long double acosl(long double x);

\vspace{2ex}

%% asin %%

\highlight{constexpr} float asin(float x); \stdcomment{library.c}

\highlight{constexpr} double asin(double x);

\highlight{constexpr} long double asin(long double x); \stdcomment{library.c}

\highlight{constexpr} float asinf(float x);

\highlight{constexpr} long double asinl(long double x);

\vspace{2ex}

%% atan %%

\highlight{constexpr} float atan(float x); \stdcomment{library.c}

\highlight{constexpr} double atan(double x);

\highlight{constexpr} long double atan(long double x); \stdcomment{library.c}

\highlight{constexpr} float atanf(float x);

\highlight{constexpr} long double atanl(long double x);

\vspace{2ex}

%% atan2 %%

\highlight{constexpr} float atan2(float y, float x); \stdcomment{library.c}

\highlight{constexpr} double atan2(double y, double x);

\highlight{constexpr} long double atan2(long double y, long double x); \stdcomment{library.c}

\highlight{constexpr} float atan2f(float y, float x);

\highlight{constexpr} long double atan2l(long double y, long double x);

\vspace{2ex}

%% cos %%

\highlight{constexpr} float cos(float x); \stdcomment{library.c}

\highlight{constexpr} double cos(double x);

\highlight{constexpr} long double cos(long double x); \stdcomment{library.c}

\highlight{constexpr} float cosf(float x);

\highlight{constexpr} long double cosl(long double x);

\vspace{2ex}

%% sin %%

\highlight{constexpr} float sin(float x); \stdcomment{library.c}

\highlight{constexpr} double sin(double x);

\highlight{constexpr} long double sin(long double x); \stdcomment{library.c}

\highlight{constexpr} float sinf(float x);

\highlight{constexpr} long double sinl(long double x);

\vspace{2ex}

%% tan %%

\highlight{constexpr} float tan(float x); \stdcomment{library.c}

\highlight{constexpr} double tan(double x);

\highlight{constexpr} long double tan(long double x); \stdcomment{library.c}

\highlight{constexpr} float tanf(float x);

\highlight{constexpr} long double tanl(long double x);

\vspace{2ex}

%% acosh %%

\highlight{constexpr} float acosh(float x); \stdcomment{library.c}

\highlight{constexpr} double acosh(double x);

\highlight{constexpr} long double acosh(long double x); \stdcomment{library.c}

\highlight{constexpr} float acoshf(float x);

\highlight{constexpr} long double acoshl(long double x);

\vspace{2ex}

%% asinh %%

\highlight{constexpr} float asinh(float x); \stdcomment{library.c}

\highlight{constexpr} double asinh(double x);

\highlight{constexpr} long double asinh(long double x); \stdcomment{library.c}

\highlight{constexpr} float asinhf(float x);

\highlight{constexpr} long double asinhl(long double x);

\vspace{2ex}

%% atanh %%

\highlight{constexpr} float atanh(float x); \stdcomment{library.c}

\highlight{constexpr} double atanh(double x);

\highlight{constexpr} long double atanh(long double x); \stdcomment{library.c}

\highlight{constexpr} float atanhf(float x);

\highlight{constexpr} long double atanhl(long double x);

\vspace{2ex}

%% cosh %%

\highlight{constexpr} float cosh(float x); \stdcomment{library.c}

\highlight{constexpr} double cosh(double x);

\highlight{constexpr} long double cosh(long double x); \stdcomment{library.c}

\highlight{constexpr} float coshf(float x);

\highlight{constexpr} long double coshl(long double x);

\vspace{2ex}

%% sinh %%

\highlight{constexpr} float sinh(float x); \stdcomment{library.c}

\highlight{constexpr} double sinh(double x);

\highlight{constexpr} long double sinh(long double x); \stdcomment{library.c}

\highlight{constexpr} float sinhf(float x);

\highlight{constexpr} long double sinhl(long double x);

\vspace{2ex}

%% tanh %%

\highlight{constexpr} float tanh(float x); \stdcomment{library.c}

\highlight{constexpr} double tanh(double x);

\highlight{constexpr} long double tanh(long double x); \stdcomment{library.c}

\highlight{constexpr} float tanhf(float x);

\highlight{constexpr} long double tanhl(long double x);

\vspace{2ex}

%% exp %%

\highlight{constexpr} float exp(float x); \stdcomment{library.c}

\highlight{constexpr} double exp(double x);

\highlight{constexpr} long double exp(long double x); \stdcomment{library.c}

\highlight{constexpr} float expf(float x);

\highlight{constexpr} long double expl(long double x);

\vspace{2ex}

%% exp2 %%

\highlight{constexpr} float exp2(float x); \stdcomment{library.c}

\highlight{constexpr} double exp2(double x);

\highlight{constexpr} long double exp2(long double x); \stdcomment{library.c}

\highlight{constexpr} float exp2f(float x);

\highlight{constexpr} long double exp2l(long double x);

\vspace{2ex}

%% expm1 %%

\highlight{constexpr} float expm1(float x); \stdcomment{library.c}

\highlight{constexpr} double expm1(double x);

\highlight{constexpr} long double expm1(long double x); \stdcomment{library.c}

\highlight{constexpr} float expm1f(float x);

\highlight{constexpr} long double expm1l(long double x);

\vspace{2ex}

%% frexp %%

\oldhighlight{constexpr}  float frexp(float value, int* exp); \stdcomment{library.c}

\oldhighlight{constexpr}  double frexp(double value, int* exp);

\oldhighlight{constexpr}  long double frexp(long double value, int* exp); \stdcomment{library.c}

\oldhighlight{constexpr}  float frexpf(float value, int* exp);

\oldhighlight{constexpr}  long double frexpl(long double value, int* exp);

\vspace{2ex}

%% ilogb %%

\oldhighlight{constexpr} int ilogb(float x); \stdcomment{library.c}

\oldhighlight{constexpr} int ilogb(double x);

\oldhighlight{constexpr} int ilogb(long double x); \stdcomment{library.c}

\oldhighlight{constexpr} int ilogbf(float x);

\oldhighlight{constexpr} int ilogbl(long double x);

\vspace{2ex}

%% ldexp %%

\oldhighlight{constexpr} float ldexp(float x, int exp); \stdcomment{library.c}

\oldhighlight{constexpr} double ldexp(double x, int exp);

\oldhighlight{constexpr} long double ldexp(long double x, int exp); \stdcomment{library.c}

\oldhighlight{constexpr} float ldexpf(float x, int exp);

\oldhighlight{constexpr} long double ldexpl(long double x, int exp);

\vspace{2ex}

%% log %%

\highlight{constexpr} float log(float x); \stdcomment{library.c}

\highlight{constexpr} double log(double x);

\highlight{constexpr} long double log(long double x); \stdcomment{library.c}

\highlight{constexpr} float logf(float x);

\highlight{constexpr} long double logl(long double x);

\vspace{2ex}

%% log 10 %%

\highlight{constexpr} float log10(float x); \stdcomment{library.c}

\highlight{constexpr} double log10(double x);

\highlight{constexpr} long double log10(long double x); \stdcomment{library.c}

\highlight{constexpr} float log10f(float x);

\highlight{constexpr} long double log10l(long double x);

\vspace{2ex}

%% loglp %%

\highlight{constexpr} float log1p(float x); \stdcomment{library.c}

\highlight{constexpr} double log1p(double x);

\highlight{constexpr}long double log1p(long double x); \stdcomment{library.c}

\highlight{constexpr} float log1pf(float x);

\highlight{constexpr} long double log1pl(long double x);

\vspace{2ex}

%% log2 %%

\highlight{constexpr} float log2(float x); \stdcomment{library.c}

\highlight{constexpr} double log2(double x);

\highlight{constexpr} long double log2(long double x); \stdcomment{library.c}

\highlight{constexpr} float log2f(float x);

\highlight{constexpr} long double log2l(long double x);

\vspace{2ex}

%% logb %%

\oldhighlight{constexpr} float logb(float x); \stdcomment{library.c}

\oldhighlight{constexpr} double logb(double x);

\oldhighlight{constexpr} long double logb(long double x); \stdcomment{library.c}

\oldhighlight{constexpr} float logbf(float x);

\oldhighlight{constexpr} long double logbl(long double x);

\vspace{2ex}

%% modf %%

\oldhighlight{constexpr}  float modf(float value, float* iptr); \stdcomment{library.c}

\oldhighlight{constexpr}  double modf(double value, double* iptr);

\oldhighlight{constexpr}  long double modf(long double value, long double* iptr); \stdcomment{library.c}

\oldhighlight{constexpr}  float modff(float value, float* iptr);

\oldhighlight{constexpr}  long double modfl(long double value, long double* iptr);

\vspace{2ex}

%% scalbn %%

\oldhighlight{constexpr} float scalbn(float x, int n); \stdcomment{library.c}

\oldhighlight{constexpr} double scalbn(double x, int n);

\oldhighlight{constexpr} long double scalbn(long double x, int n); \stdcomment{library.c}

\oldhighlight{constexpr} float scalbnf(float x, int n);

\oldhighlight{constexpr} long double scalbnl(long double x, int n);

\vspace{2ex}

%% scalbln %%

\oldhighlight{constexpr} float scalbln(float x, long int n); \stdcomment{library.c}

\oldhighlight{constexpr} double scalbln(double x, long int n);

\oldhighlight{constexpr} long double scalbln(long double x, long int n); \stdcomment{library.c}

\oldhighlight{constexpr} float scalblnf(float x, long int n);

\oldhighlight{constexpr} long double scalblnl(long double x, long int n);

\vspace{2ex}

\highlight{constexpr} float cbrt(float x); \stdcomment{library.c}

\highlight{constexpr} double cbrt(double x);

\highlight{constexpr} long double cbrt(long double x); \stdcomment{library.c}

\highlight{constexpr} float cbrtf(float x);

\highlight{constexpr} long double cbrtl(long double x);

\vspace{2ex}

//  [c.math.abs], {\it absolute values}

%% abs %%

\oldhighlight{constexpr} int abs(int j);

\oldhighlight{constexpr} long int abs(long int j);

\oldhighlight{constexpr} long long int abs(long long int j);

\oldhighlight{constexpr} float abs(float j);

\oldhighlight{constexpr} double abs(double j);

\oldhighlight{constexpr}long double abs(long double j);

\vspace{2ex}

%% fabs %%

\oldhighlight{constexpr} float fabs(float x); \stdcomment{library.c}

\oldhighlight{constexpr} double fabs(double x);

\oldhighlight{constexpr} long double fabs(long double x); \stdcomment{library.c}

\oldhighlight{constexpr} float fabsf(float x);

\oldhighlight{constexpr} long double fabsl(long double x);

\vspace{2ex}

%% hypot %%

\highlight{constexpr} float hypot(float x, float y); \stdcomment{library.c}

\highlight{constexpr} double hypot(double x, double y);

\highlight{constexpr} long double hypot(long double x, long double y); \stdcomment{library.c}

\highlight{constexpr} float hypotf(float x, float y);

\highlight{constexpr} long double hypotl(long double x, long double y);

\vspace{2ex}

// [c.math.hypot3], {\it three-dimensional hypotenuse}

%% hypot3 5%

\highlight{constexpr} float hypot(float x, float y, float z);

\highlight{constexpr} double hypot(double x, double y, double z);

\highlight{constexpr} long double hypot(long double x, long double y, long double z);

%% pow %%
\vspace{2ex}

\highlight{constexpr} float pow(float x, float y); \stdcomment{library.c}

\highlight{constexpr} double pow(double x, double y);

\highlight{constexpr} long double pow(double x, double y); \stdcomment{library.c}

\highlight{constexpr} float powf(float x, float y);

\highlight{constexpr} long double powl(long double x, long double y);

\vspace{2ex}

%% sqrt %%

\highlight{constexpr} float sqrt(float x); \stdcomment{library.c}

\highlight{constexpr} double sqrt(double x);

\highlight{constexpr} long double sqrt(double x); \stdcomment{library.c}

\highlight{constexpr} float sqrtf(float x);

\highlight{constexpr} long double sqrtl(long double x);

\vspace{2ex}

%% erf %%

\highlight{constexpr} float erf(float x); \stdcomment{library.c}

\highlight{constexpr} double erf(double x);

\highlight{constexpr} long double erf(long double x); \stdcomment{library.c}

\highlight{constexpr} float erff(float x);

\highlight{constexpr} long double erfl(long double x);

\vspace{2ex}
%% erfc %%

\highlight{constexpr} float erfc(float x); \stdcomment{library.c}

\highlight{constexpr} double erfc(double x);

\highlight{constexpr} long double erfc(long double x); \stdcomment{library.c}

\highlight{constexpr} float erfcf(float x);

\highlight{constexpr} long double erfcl(long double x);

\vspace{2ex}

%% lgamma %%

\highlight{constexpr} float lgamma(float x); \stdcomment{library.c}

\highlight{constexpr} double lgamma(double x);

\highlight{constexpr} long double lgamma(long double x); \stdcomment{library.c}

\highlight{constexpr} float lgammaf(float x);

\highlight{constexpr} long double lgammal(long double x);

\vspace{2ex}
%% tgamma %%

\highlight{constexpr} float tgamma(float x); \stdcomment{library.c}

\highlight{constexpr} double tgamma(double x);

\highlight{constexpr} long double tgamma(long double x); \stdcomment{library.c}

\highlight{constexpr} float tgammaf(float x);

\highlight{constexpr} long double tgammal(long double x);

\vspace{2ex}
%% ceil %%

\oldhighlight{constexpr} float ceil(float x); \stdcomment{library.c}

\oldhighlight{constexpr} double ceil(double x);

\oldhighlight{constexpr} long double ceil(long double x); \stdcomment{library.c}

\oldhighlight{constexpr} float ceilf(float x);

\oldhighlight{constexpr} long double ceill(long double x);

\vspace{2ex}

%% floor %%

\oldhighlight{constexpr} float floor(float x); \stdcomment{library.c}

\oldhighlight{constexpr} double floor(double x);

\oldhighlight{constexpr} long double floor(long double x); \stdcomment{library.c}

\oldhighlight{constexpr} float floorf(float x);

\oldhighlight{constexpr} long double floorl(long double x);

\vspace{2ex}

%% nearbyint %%

float nearbyint(float x); \stdcomment{library.c}

double nearbyint(double x);

long double nearbyint(long double x); \stdcomment{library.c}

 float nearbyintf(float x);

 long double nearbyintl(long double x);

\vspace{2ex}

%% rint %%

 float rint(float x); \stdcomment{library.c}

 double rint(double x);

long double rint(long double x); \stdcomment{library.c}

 float rintf(float x);

 long double rintl(long double x);

\vspace{2ex}

%% lrint %%

long int lrint(float x); \stdcomment{library.c}

 long int lrint(double x);

long int lrint(long double x); \stdcomment{library.c}

 long int lrintf(float x);

long int lrintl(long double x);

\vspace{2ex}

%% llrint %%

long long int llrint(float x); \stdcomment{library.c}

long long int llrint(double x);

long long int llrint(long double x); \stdcomment{library.c}

long long int llrintf(float x);

long long int llrintl(long double x);

\vspace{2ex}

%% round %%

\oldhighlight{constexpr} float round(float x); \stdcomment{library.c}

\oldhighlight{constexpr} double round(double x);

\oldhighlight{constexpr} long double round(long double x); \stdcomment{library.c}

\oldhighlight{constexpr} float roundf(float x);

\oldhighlight{constexpr} long double roundl(long double x);

\vspace{2ex}

%% lround %%

\oldhighlight{constexpr} long int lround(float x); \stdcomment{library.c}

\oldhighlight{constexpr} long int lround(double x);

\oldhighlight{constexpr} long int lround(long double x); \stdcomment{library.c}

\oldhighlight{constexpr} long int lroundf(float x);

\oldhighlight{constexpr} long int lroundl(long double x);

\vspace{2ex}

%% llround %%

\oldhighlight{constexpr} long long int llround(float x); \stdcomment{library.c}

\oldhighlight{constexpr} long long int llround(double x);

\oldhighlight{constexpr} long long int llround(long double x); \stdcomment{library.c}

\oldhighlight{constexpr} long long int llroundf(float x);

\oldhighlight{constexpr} long long int llroundl(long double x);

\vspace{2ex}

%% trunc %%

\oldhighlight{constexpr} float trunc(float x); \stdcomment{library.c}

\oldhighlight{constexpr} double trunc(double x);

\oldhighlight{constexpr} long double trunc(long double x); \stdcomment{library.c}

\oldhighlight{constexpr} float truncf(float x);

\oldhighlight{constexpr} long double truncl(long double x);

\vspace{2ex}

%% fmod %%

\oldhighlight{constexpr} float fmod(float x, float y); \stdcomment{library.c}

\oldhighlight{constexpr} double fmod(double x, double y);

\oldhighlight{constexpr} long double fmod(long double x, long double y); \stdcomment{library.c}

\oldhighlight{constexpr} float fmodf(float x, float y);

\oldhighlight{constexpr} long double fmodl(long double x, long double y);

\vspace{2ex}

%% remainder %%

\oldhighlight{constexpr} float remainder(float x, float y); \stdcomment{library.c}

\oldhighlight{constexpr} double remainder(double x, double y);

\oldhighlight{constexpr} long double remainder(long double x, long double y); \stdcomment{library.c}

\oldhighlight{constexpr} float remainderf(float x, float y);

\oldhighlight{constexpr} long double remainderl(long double x, long double y);

\vspace{2ex}

%% remquo %%

\oldhighlight{constexpr}  float remquo(float x, float y, int* quo); \stdcomment{library.c}

\oldhighlight{constexpr}  double remquo(double x, double y, int* quo);

\oldhighlight{constexpr}  long double remquo(long double x, long double y, int* quo); \stdcomment{library.c}

\oldhighlight{constexpr}  float remquof(float x, float y, int* quo);

\oldhighlight{constexpr}  long double remquol(long double x, long double y, int* quo);

\vspace{2ex}

%% copysign %%

\oldhighlight{constexpr} float copysign(float x, float y); \stdcomment{library.c}

\oldhighlight{constexpr} double copysign(double x, double y);

\oldhighlight{constexpr} long double copysign(long double x, long double y); \stdcomment{library.c}

\oldhighlight{constexpr} float copysignf(float x, float y);

\oldhighlight{constexpr} long double copysignl(long double x, long double y);

\vspace{2ex}

double nan(const char* tagp);

float nanf(const char* tagp);

long double nanl(const char* tagp);

\vspace{2ex}

%% nextafter %%

\oldhighlight{constexpr}  float nextafter(float x, float y); \stdcomment{library.c}

\oldhighlight{constexpr}  double nextafter(double x, double y);

\oldhighlight{constexpr}  long double nextafter(long double x, long double y); \stdcomment{library.c}

\oldhighlight{constexpr}  float nextafterf(float x, float y);

\oldhighlight{constexpr}  long double nextafterl(long double x, long double y);

\vspace{2ex}

%% nexttoward %%

\oldhighlight{constexpr}  float nexttoward(float x, long double y); \stdcomment{library.c}

\oldhighlight{constexpr}  double nexttoward(double x, long double y);

\oldhighlight{constexpr}  long double nexttoward(long double x, long double y); \stdcomment{library.c}

\oldhighlight{constexpr}  float nexttowardf(float x, long double y);

\oldhighlight{constexpr}  long double nexttowardl(long double x, long double y);

\vspace{2ex}

%% fdim %%

\oldhighlight{constexpr}  float fdim(float x, float y); \stdcomment{library.c}

\oldhighlight{constexpr}  double fdim(double x, double y);

\oldhighlight{constexpr}  long double fdim(long double x, long double y); \stdcomment{library.c}

\oldhighlight{constexpr}  float fdimf(float x, float y);

\oldhighlight{constexpr}  long double fdiml(long double x, long double y);

\vspace{2ex}

%% fmax %%

\oldhighlight{constexpr}  float fmax(float x, float y); \stdcomment{library.c}

\oldhighlight{constexpr}  double fmax(double x, double y);

\oldhighlight{constexpr}  long double fmax(long double x, long double y); \stdcomment{library.c}

\oldhighlight{constexpr}  float fmaxf(float x, float y);

\oldhighlight{constexpr}  long double fmaxl(long double x, long double y);

\vspace{2ex}

%% fmin %%

\oldhighlight{constexpr} float fmin(float x, float y); \stdcomment{library.c}

\oldhighlight{constexpr}  double fmin(double x, double y);

\oldhighlight{constexpr}  long double fmin(long double x, long double y); \stdcomment{library.c}

\oldhighlight{constexpr}  float fminf(float x, float y);

\oldhighlight{constexpr} long double fminl(long double x, long double y);

\vspace{2ex}

%% fma %%

\oldhighlight{constexpr}  float fma(float x, float y, float z); \stdcomment{library.c}

\oldhighlight{constexpr}  double fma(double x, double y, double z);

\oldhighlight{constexpr}  long double fma(long double x, long double y, long double z); \stdcomment{library.c}

\oldhighlight{constexpr}  float fmaf(float x, float y, float z);

\oldhighlight{constexpr}  long double fmal(long double x, long double y, long double z);

\vspace{2ex}

// [c.math.fpclass], {\it classification / comparison functions:}

%% fpclassify %%

\oldhighlight{constexpr} int fpclassify(float x);

\oldhighlight{constexpr} int fpclassify(double x);

\oldhighlight{constexpr} int fpclassify(long double x);

\vspace{2ex}

%% isinfinite %%

\oldhighlight{constexpr} int isfinite(float x);

\oldhighlight{constexpr} int isfinite(double x);

\oldhighlight{constexpr} int isfinite(long double x);

\vspace{2ex}

%% isinf %%

\oldhighlight{constexpr} int isinf(float x);

\oldhighlight{constexpr} int isinf(double x);

\oldhighlight{constexpr} int isinf(long double x);

\vspace{2ex}

%% isnan %%

\oldhighlight{constexpr} int isnan(float x);

\oldhighlight{constexpr} int isnan(double x);

\oldhighlight{constexpr} int isnan(long double x);

\vspace{2ex}

%% isnormal %%

\oldhighlight{constexpr} int isnormal(float x);

\oldhighlight{constexpr} int isnormal(double x);

\oldhighlight{constexpr} int isnormal(long double x);

\vspace{2ex}

%% signbit %%

\oldhighlight{constexpr} int signbit(float x);

\oldhighlight{constexpr} int signbit(double x);

\oldhighlight{constexpr} int signbit(long double x);

\vspace{2ex}

%% isgreater %%

\oldhighlight{constexpr} int isgreater(float x, float y);

\oldhighlight{constexpr} int isgreater(double x, double y);

\oldhighlight{constexpr} int isgreater(long double x, long double y);

\vspace{2ex}

%% isgreaterequal %%

\oldhighlight{constexpr} int isgreaterequal(float x, float y);

\oldhighlight{constexpr} int isgreaterequal(double x, double y);

\oldhighlight{constexpr} int isgreaterequal(long double x, long double y);

\vspace{2ex}

%% isless %%

\oldhighlight{constexpr} int isless(float x, float y);

\oldhighlight{constexpr} int isless(double x, double y);

\oldhighlight{constexpr} int isless(long double x, long double y);

\vspace{2ex}

%% islessequal %%

\oldhighlight{constexpr} int islessequal(float x, float y);

\oldhighlight{constexpr} int islessequal(double x, double y);

\oldhighlight{constexpr} int islessequal(long double x, long double y);

\vspace{2ex}

%% islessgreater %%

\oldhighlight{constexpr} int islessgreater(float x, float y);

\oldhighlight{constexpr} int islessgreater(double x, double y);

\oldhighlight{constexpr} int islessgreater(long double x, long double y);

\vspace{2ex}

%% isunordered %%

\oldhighlight{constexpr} int isunordered(float x, float y);

\oldhighlight{constexpr} int isunordered(double x, double y);

\oldhighlight{constexpr} int isunordered(long double x, long double y);

}


\subsection{Modifications to ``Three-dimensional hypotenuse''  [c.math.hpot3]}

\code{
	\highlight{constexpr} float hypot(float x, float y); 

	\highlight{constexpr} double hypot(double x, double y);

	\highlight{constexpr} long double hypot(double x, double y); 
}

\end{document}
