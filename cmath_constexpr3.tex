\documentclass[prd,twocolumn,amsmath,amssymb,nofootinbib,eqsecnum]{revtex4-1}

\usepackage{xspace} %Dude this will blow your mind!
\usepackage{amsthm}
\usepackage{color}
\usepackage{soul}

\newcommand{\constexpr}{\code{constexpr}\xspace}
\newcommand{\code}[1]{{\tt #1}}
\newcommand{\header}[1]{{\tt <#1>}}
\newcommand{\cmath}{\header{cmath}}
\newcommand{\cstdlib}{\header{cstdlib}}

\newcommand{\FEINVALID}{{\tt FE\_INVALID}}
\newcommand{\FEDIVBYZERO}{{\tt FE\_DIVBYZERO}}
\newcommand{\FEINEXACT}{{\tt FE\_INEXACT}}
\newcommand{\FEUNDERFLOW}{{\tt FE\_UNDERFLOW}}
\newcommand{\FEOVERFLOW}{{\tt FE\_OVERFLOW}}
\newcommand{\FLTRADIX}{{\tt FLT\_RADIX}}
\newcommand{\Operators}{\ensuremath{+,-,\times,/}}

\newcommand{\highlight}[1]{{\color{red} #1}}
\newcommand{\stdcomment}[1]{{// {\em see} [#1]}}

\newcommand{\eq}[1]{(\ref{eq:#1})}

\newtheorem*{proposal*}{Proposal}

\makeatletter
\def\l@subsection#1#2{}
\def\l@subsubsection#1#2{}
\makeatother

\def\bibsection{\section*{\refname}} 

\begin{document}


\title{\constexpr for \cmath\ and \cstdlib}
%\author{Edward J.~Rosten \& Oliver J.~Rosten}
%\date{\today}

\author{
\hspace{11.5em}
\begin{tabular}{ll}
	Document: & D0533R1
\\
	Date: & \today
\\
	Project: & Programming Language C++, Library Working Group
\\
	Audience: & SG6, LEWG
\\
	Reply to: & Edward J.\ Rosten / Oliver J.\ Rosten 
	{\tt $<$forename.rosten@gmail.com$>$}
\end{tabular}
}

\begin{abstract}

\begin{center} {\bf Abstract} \end{center}
We propose simple criteria for selecting functions in \cmath\ which should be
declared \constexpr. There is a small degree of overlap with \cstdlib.
 The aim is to transparently select a sufficiently large portion of \cmath\ 
in order to be useful but without placing too much burden on compiler vendors.
	
\end{abstract}


\maketitle

\tableofcontents

\section{Introduction}

This paper seeks to rectify the current absence of \constexpr in
\cmath\ (and also in \cstdlib), so as to broaden the range of numeric computations that can be
performed using standard library facilities. While in principle almost every function
in \cmath\ could be declared \constexpr, we strike a balance between coverage and onus on compiler/library vendors.


\section{Motivation \& Scope}

The introduction of \constexpr has facilitated intuitive compile-time
programming. However, not a single function in \cmath\ is currently declared \constexpr,
thereby artificially restricting what can be done at compile-time within the standard library.
Nevertheless, from casual inspection of \cmath, it may
not be immediately obvious precisely which functions should be declared
\constexpr. In this paper, we seek an organizing principle which selects functions
which are in a sense no more complicated than the elementary arithmetic operations
(\Operators). This is justified since the latter already support \constexpr.

Indeed, two subtleties can be resolved by appealing to the fact that they must be dealt with in implementing \constexpr for the arithmetic operators. In particular, various functions in \cmath\ may set global flags and/or depend on the rounding mode. These issues are discussed in the next two subsections, following which the concrete statement of the conditions under which a function should be declared \constexpr is given.

\subsection{Global Flags}

Under certain conditions, various functions in \cmath\ may set global flags.
Specifically, \code{errno} may be set and/or the various floating-point
exception flags, \FEDIVBYZERO, \FEINVALID, \FEOVERFLOW, \FEUNDERFLOW\ and
\FEINEXACT\ may be raised. 

For example, \code{std::round(double x)}, which rounds its argument to the nearest integer value, raises \FEINVALID\ in the case that its argument is NaN or $\pm \infty$. This may seem problematic if one wishes to declare \code{std::round(double x)} to be \constexpr. 
However, the issue of raising exception flags in a \constexpr context is nothing new: it is already faced by
the standard arithmetic operators. Nevertheless, the latter are available for use in constant
expressions. The proposed strategy is to mimic the behaviour of the arithmetic
operators.

To be precise, functions declared \constexpr, \emph{when used in a \constexpr
context}, should give a compiler error if division by zero, domain errors or
overflows occur. When not used in a \constexpr context, the various global
flags should be set as normal. This distinction between these two contexts
implies that any implementation cannot be done as a pure library extension.
However, below
we will introduce a criterion which restricts the proposed set of \constexpr\ functions 
to those which are, in a sense, simple. Consequently, while there will be some burden 
on compiler vendors it should be minimal. 

\subsection{Rounding Mode}

Some of the functions in \cmath\ depend on the rounding mode, which is something which may be changed at runtime. To facilitate the discussion, we wish to distinguish two situations, which we will call \emph{weak}/\emph{strong} rounding mode dependence. 

Weak dependence is that already experienced by the arithmetic operators. For example, consider 10.0/3.0: the result depends on the rounding mode. We refer to this rounding mode dependence as weak since it is an artefact of the limited precision of floating-point numbers. However, it is perfectly legitimate to declare
\begin{equation}
	\mbox{\code{constexpr double x\{10.0/3.0\}}}.
\label{eq:10/3}
\end{equation}
Therefore, when deciding which functions in \cmath\ should be \constexpr, we will not rule out functions with weak rounding mode dependence.
As for~\eq{10/3}, what result should we expect? According to [cfenv.syn] footnote 1, the result is implementation defined. However, this issue is currently under active discussion. 

The key point for this paper is that, whatever decision is made, the approach can be consistently applied to those functions in \cmath\ which we propose should be declared \constexpr. It is worth noting that the number of functions in this proposal which are dependent on the rounding mode is rather small (see \ref{sec:dd}).

Having dealt with weak rounding mode dependence, consider \code{float nearbyint(float x)}. This function rounds its argument to the nearest integer \emph{taking account of the current rounding mode}. Thus, a change to the rounding mode can change the answer by unity. This dependence on the rounding mode is not an artefact of limited precision and hence we call it strong.

In this proposal, we chose to exclude functions with strong rounding mode dependence from being declared \constexpr. This respects the fact that these functions are explicitly designed to depend on the runtime environment.


\subsection{Conditions for \constexpr}

Taking into account the above consideration, we propose the following in order to put
the application of \constexpr on a rigorous footing:
\begin{proposal*}
	A function in \cmath\ shall be declared \constexpr if and only if:
	\begin{enumerate}
		\item When taken to act on the set of rational numbers, the function is closed (excluding division 
		by zero);
		
		\item The function does not modify any of its arguments which have external visibility;
		
		\item The function is not strongly dependent on the rounding mode.
	\end{enumerate}
\end{proposal*}

By means of a brief illustration, $\code{abs}$ satisfies all three criteria; however, functions such as 
\code{exp}, \code{sqrt}, \code{cos}, \code{sin} fall foul of the first criterion and so are excluded as 
\constexpr candidates. Finally, as discussed above, \code{nearbyint} fails the final criterion.


\section{State of the Art}

Both GCC and clang already support \constexpr within \cmath\ to varying extents. 
Indeed, GCC 5.3.0 declares all functions, with the exception of those taking a pointer argument
(cf.\ the second criterion), as \constexpr. Therefore, an implementation of the changes to the standard proposed in this paper is available (indeed, this implementation goes beyond what we propose).
While clang does not go nearly as far as GCC, it does offer some functions as 
builtins and is able to use them to perform compile time computations, constant
propagation and so on. It is therefore hoped that any burden on compiler vendors implicit in this
proposal is minimal.

\section{Impact On the Standard}

This proposal adds an extra requirement to [expr.const] pertaining to the definition of what constitutes a constant expression. In particular, if a mathematical function encounters a domain error or overflow, then it may not form part of a constant expression. As such, this proposal does not amount to a pure library extension. Beyond this addition, the proposed changes amount to scattering \constexpr\ throughout the existing headers \cmath\ and \header{cstdlib}, according to the rules specified earlier. 

In this proposal, we have chosen for the standard to remain silent on the issue of the interaction of rounding mode dependence with constant expressions. On the one hand, this is no worse than the current situation regarding the arithmetic operators. On the other, the active discussion about how to optimally resolve this matter suggests to us that the issue is better served by a separate proposal.

\section{Design Decisions}
\label{sec:dd}

There are several obvious candidates in \header{cmath} to which \constexpr should be applied, such as \code{abs}, \code{floor}, \code{ceil}.  But beyond this, it is desirable to
apply \constexpr throughout \cmath\ in a consistent manner. Ideally, this consistency of approach should encompass prior work on \code{complex}, since it has already been proposed that, in addition to the arithmetic operations, \code{complex::norm} and a few other functions be declared \constexpr~\cite{AP-complex}. For this purpose, ideally one would
like one or more criteria rooted in mathematics. However, the reality is that
this is insufficiently restrictive. Nevertheless, mathematics was our starting
point; as such, any condition on functions to be \constexpr must select the
basic arithmetic operations, $(+,-,\times,/)$, since these may already be used in a 
\constexpr\ context.

Mathematically, a field is closed under the elementary operations of addition
and multiplication.
Numeric types do not form a field; however, since the basic arithmetic
operations are already declared \constexpr, this suggests that it may be
possible to utilize a field which captures enough of the properties of numeric
types in order to be useful in formulating criteria for the application of
\constexpr. The set of rational numbers is the natural candidate since all valid
values of numeric types are elements of this set and, moreover, the rationals close over
$(+,-,\times,/)$ (with zero excluded for division).

The subtlety of global flags being set upon encountering floating-point
exceptions presents a challenge. If all functions which can set such flags are
excluded from the list to tag \constexpr, then the remaining list is rather
sparse. To achieve something more useful suggests expanding the set to include
those functions which are `simple enough'. These considerations lead to the
first condition of the proposal.
Tables~\ref{tab:26.9.1}--\ref{tab:26.9.4} contain the functions in \cmath\
satisfying this criterion and indicate whether or not they pass the second
and third criteria as well. 

To reduce space, the following convention is observed. The
functions listed in [c.math] are divided into blocks of closely related
functions such as those shown in table~\ref{tab:example}. 
\begin{table}[h]
	\begin{tabular}{l}
		\code{int ilogb(float arg)}
	\\
		\code{int ilogb(double arg)}
	\\
		\code{int ilogb(long double arg)}
	\\
		\code{int ilogbf(float arg)}
	\\
		\code{int ilogbl(long double arg)}
	\end{tabular}
\caption{Example of a family of functions which appear as a block in the standard.}
\label{tab:example}
\end{table}
Note that while the first three functions are overloads, the fourth and fifth
have differing names.  When classifying those functions which satisfy the first
criterion, we will present just the first function in each such block, with the
understanding that the others are similar in this regard. Furthermore, 
we supply various comments in the
third column of the tables, observing the following shorthands:
\begin{enumerate}
	\item G: May set global variable;
	
	\item V: Modifies argument with external visibility;
	
	\item S: Depends strongly on rounding mode;
	
	\item W: Depends weakly on the rounding mode;
	
	\item w: Depends weakly on the rounding mode only if \FLTRADIX\ is not 2;
	
	\item U: Depends weakly on the rounding mode only in the case of underflow.
\end{enumerate}
If more than one of these applies, then this is indicated using $\vert$; for example, if a function may set a global variable and also depends on the rounding mode, this would be indicated by G$\vert$S. Finally, implementation dependence is denoted by a $\star$ so that, for example, G$\star$ means that whether or not a global variable may be set depends on the implementation.
\begin{table}[h]
	\begin{tabular}{lcc}
		Function & Pass & Comment
	\\
	\hline \hline
		\code{float frexp(float value, int* exp)} & No & V$\vert$w
	\\
	\hline
		\code{int ilogb(float arg)} & Yes & G
	\\
	\hline
		\code{float ldexp(float x, int exp)} & Yes & G$\vert$w		
	\\
	\hline
		\code{float logb(float arg)} & Yes & G
	\\
	\hline
		\code{float modf(float value, float* iptr)} & No & V
	\\
	\hline
		\code {float scalbn(float x, int n)} & Yes & G$\vert$U
	\\
	\hline
		\code {float scalbln(float x, long int n)} & Yes & G$\vert$U
	\end{tabular}
\caption{Various functions declared in [cmath.syn] which close on the rationals.}
\label{tab:26.9.1}
\end{table}


\begin{table}[h]
	\begin{tabular}{lc}
		Function & Pass
	\\
	\hline \hline
		\code{int abs(int j)} & Yes
	\\
	\hline
		\code{float fabs(float x)} & Yes
	\end{tabular}
\caption{Absolute values declared in [cmath.syn]  which close on the rationals.}
\label{tab:26.9.2}
\end{table}

\begin{table}[h]
	\begin{tabular}{lcc}
		Function & Pass & Comment
	\\
	\hline \hline
		\code{float ceil(float x)} &  Yes & G$\star$
	\\
	\hline
		\code{float floor(float x)} & Yes & G$\star$
	\\
	\hline
		\code{float nearbyint(float x)} & No & S
	\\
	\hline
		\code{float rint(float x)} & No & G$\vert$S
	\\
	\hline
		\code{long int lrint(float x)} &  No & G$\vert$S
	\\
	\hline
		\code{long long int llrint(float x)} & No & G$\vert$S
	\\
	\hline
		\code{float round(float x)} & Yes & G
	\\
	\hline
		\code{float lround(float x)} & Yes & G
	\\
	\hline
		\code{float llround(float x)} & Yes & G
	\\
	\hline
		\code{float trunc(float x)} &  Yes & G
	\\
	\hline
		\code{float fmod(float x, float y)} & Yes & G$\vert$W
	\\
	\hline
		\code{float remainder(float x, float y)} & Yes & G$\vert$W$\star$
	\\
	\hline
		\code{float remquo(float x, float y, int* quo)} & No & V$\vert$G$\vert$W$\star$
	\\
	\hline
		\code{float copysign(float x, float y)} & Yes &
	\\
	\hline
		\code{float nextafter(float x, float y)} & Yes & G
	\\
	\hline
		\code{float nexttoward(float x, long double y)} & Yes & G
	\\
	\hline
		\code{float fdim(float x, float y)} & Yes & G$\vert$U
	\\
	\hline
		\code{float fmax(float x, float y)} & Yes &
	\\
	\hline
		\code{float fmin(float x, float y)} & Yes &
	\\
	\hline
		\code{float fma(float x, float y, float z)} & Yes & G$\vert$W
	\end{tabular}
\caption{Additional functions declared in [cmath.syn]  which close on the rationals.}
\label{tab:26.9.3}
\end{table}

\begin{table}[h]
	\begin{tabular}{lcc}
		Function & Pass & Comment
	\\
	\hline \hline
		\code{int fpclassify(float x);} & Yes &
	\\
	\hline
		\code{int isfinite(float x)} & Yes &
	\\
	\hline
		\code{int isinf(float x)} & Yes & $\dagger$
	\\
	\hline
		\code{int isnan(float x)} & Yes & $\dagger$ 
	\\
	\hline
		\code{int isnormal(float x)} & Yes &
	\\
	\hline
		\code{int signbit(float x)} & Yes &
	\\
	\hline
		\code{int isgreater(float x, float y)} & Yes &
	\\
	\hline
		\code{int isgreaterequal(float x, float y)} & Yes &
	\\
	\hline
		\code{int isless(float x, float y)} & Yes &
	\\
	\hline
		\code{int islessequal(float x, float y)} & Yes &
	\\
	\hline
		\code{int islessgreater(float x, float y)} & Yes &
	\\
	\hline
		\code{int isunordered(float x, float y)} & Yes &
	\end{tabular}
\caption{Comparison operators belonging to [cmath.syn] which close on the rationals. $\dagger$ --- no utility being declared \constexpr in of itself, but should be tagged \constexpr so that it can be incorporated into \constexpr functions since the latter may be called in  non-\constexpr contexts.}
\label{tab:26.9.4}
\end{table}

\section{Future Directions}

Ultimately, it is desirable to follows GCC's lead and to declare almost all functions in \cmath\ as \constexpr. This will amount to removing the first criterion of our proposal which, particularly once the issue of weak rounding mode dependence has been resolved, should hopefully be relatively uncontroversial.

\section{Revision History}

Revision 1: Includes discussion of rounding mode and future directions.

\begin{acknowledgments}
	We would like to thank Walter E.~Brown, Daniel Kr\"ugler, Antony Polukhin and Richard Smith for 
	encouragement and advice.
\end{acknowledgments}


\begin{thebibliography}{1}
	\bibitem[P0415R0]{AP-complex} Antony Polukhin, Constexpr for std::complex.
	
	\bibitem[N4687]{WorkingPaper} Richard Smith, ed., Working Draft, Standard for Programming Language C++.
\end{thebibliography}

\newpage

%\begin{widetext}
\onecolumngrid

\section{Proposed Wording}

The following proposed changes refer to the Working Paper~\cite{WorkingPaper}.

\subsection{Modification to ``Constant expressions'' [expr.const]}

\setlength{\parindent}{0pt}

(2.19) --- a relational (5.9) or equality (5.10) operator where the result is unspecified; \highlight{\st{or}}

{\small (2.20)} --- a \emph{throw-expression} (5.17)\highlight{\st{.};or}

\highlight{{\small (2.21)} --- an invocation of a mathematical function in the standard library that encounters a domain error or overflow.}


\subsection{Modifications to ``Header \header{cstdlib} synopsis'' [cstdlib.syn]}

\code{namespace std\{

\vspace{2ex}

\ldots

\vspace{2ex}

\highlight{constexpr} int abs(int j);

\highlight{constexpr}  long int abs(long int j);

\highlight{constexpr}  long long int abs(long long int j);

\highlight{constexpr}  float abs(float j);

\highlight{constexpr} double abs(double j);

\highlight{constexpr} long double abs(long double j);

\vspace{2ex}

\highlight{constexpr} long int labs(long int j);

\highlight{constexpr}  long long int llabs(long long int j);

\vspace{2ex}

\highlight{constexpr} div\_t div(int numer, int denom);

\highlight{constexpr} ldiv\_t div(long int numer, long int denom); \stdcomment{library.c}

\highlight{constexpr} lldiv\_t div(long long int numer, long long int denom); \stdcomment{library.c}

\highlight{constexpr} ldiv\_t ldiv(long int numer, long int denom);

\highlight{constexpr} lldiv\_t lldiv(long long int numer, long long int denom);	

\}}

\subsection{Modifications to  ``Header \header{cmath} synopsis'' [cmath.syn]}

\code{
\ldots
\vspace{2ex}

namespace std\{

\vspace{2ex}
\ldots
\vspace{2ex}

float acos(float x); \stdcomment{library.c}

double acos(double x);

long double acos(long double x); \stdcomment{library.c}

float acosf(float x);

long double acosl(long double x);

\vspace{2ex}
\ldots
\vspace{2ex}

float frexp(float value, int* exp); \stdcomment{library.c}

double frexp(double value, int* exp);

long double frexp(long double value, int* exp); \stdcomment{library.c}

float frexpf(float value, int* exp);

long double frexpl(long double value, int* exp);

\vspace{2ex}

\highlight{constexpr} int ilogb(float x); \stdcomment{library.c}

\highlight{constexpr} int ilogb(double x);

\highlight{constexpr} int ilogb(long double x); \stdcomment{library.c}

\highlight{constexpr} int ilogbf(float x);

\highlight{constexpr} int ilogbl(long double x);

\vspace{2ex}

\highlight{constexpr} float ldexp(float x, int exp); \stdcomment{library.c}

\highlight{constexpr} double ldexp(double x, int exp);

\highlight{constexpr} long double ldexp(long double x, int exp); \stdcomment{library.c}

\highlight{constexpr} float ldexpf(float x, int exp);

\highlight{constexpr} long double ldexpl(long double x, int exp);

\vspace{2ex}

float log(float x); \stdcomment{library.c}

double log(double x);

long double log(long double x); \stdcomment{library.c}

float logf(float x);

long double logl(long double x);

\vspace{2ex}

float log10(float x); \stdcomment{library.c}

double log10(double x);

long double log10(long double x); \stdcomment{library.c}

float log10f(float x);

long double log10l(long double x);

\vspace{2ex}

float log1p(float x); \stdcomment{library.c}

double log1p(double x);

long double log1p(long double x); \stdcomment{library.c}

float log1pf(float x);

long double log1pl(long double x);

\vspace{2ex}

float log2(float x); \stdcomment{library.c}

double log2(double x);

long double log2(long double x); \stdcomment{library.c}

float log2f(float x);

long double log2l(long double x);

\vspace{2ex}

\highlight{constexpr} float logb(float x); \stdcomment{library.c}

\highlight{constexpr} double logb(double x);

\highlight{constexpr} long double logb(long double x); \stdcomment{library.c}

\highlight{constexpr} float logbf(float x);

\highlight{constexpr} long double logbl(long double x);

\vspace{2ex}

float modf(float value, float* iptr); \stdcomment{library.c}

double modf(double value, double* iptr);

long double modf(long double value, long double* iptr); \stdcomment{library.c}

float modff(float value, float* iptr);

long double modfl(long double value, long double* iptr);

\vspace{2ex}

\highlight{constexpr} float scalbn(float x, int n); \stdcomment{library.c}

\highlight{constexpr} double scalbn(double x, int n);

\highlight{constexpr} long double scalbn(long double x, int n); \stdcomment{library.c}

\highlight{constexpr} float scalbnf(float x, int n);

\highlight{constexpr} long double scalbnl(long double x, int n);

\vspace{2ex}

\highlight{constexpr} float scalbln(float x, long int n); \stdcomment{library.c}

\highlight{constexpr} double scalbln(double x, long int n);

\highlight{constexpr} long double scalbln(long double x, long int n); \stdcomment{library.c}

\highlight{constexpr} float scalblnf(float x, long int n);

\highlight{constexpr} long double scalblnl(long double x, long int n);

\vspace{2ex}

float cbrt(float x); \stdcomment{library.c}

double cbrt(double x);

long double cbrt(long double x); \stdcomment{library.c}

float cbrtf(float x);

long double cbrtl(long double x);

\vspace{2ex}

//  [c.math.abs], {\em absolute values}

\highlight{constexpr} int abs(int j);

\highlight{constexpr} long int abs(long int j);

\highlight{constexpr} long long int abs(long long int j);

\highlight{constexpr} float abs(float j);

\highlight{constexpr} double abs(double j);

\highlight{constexpr}long double abs(long double j);

\vspace{2ex}

\highlight{constexpr} float fabs(float x); \stdcomment{library.c}

\highlight{constexpr} double fabs(double x);

\highlight{constexpr} long double fabs(long double x); \stdcomment{library.c}

\highlight{constexpr} float fabsf(float x);

\highlight{constexpr} long double fabsl(long double x);

\vspace{2ex}

float hypot(float x, float y); \stdcomment{library.c}

double hypot(double x, double y);

long double hypot(double x, double y); \stdcomment{library.c}

float hypotf(float x, float y);

long double hypotl(long double x, long double y);

\vspace{2ex}

// [c.math.hypot3], {\em three-dimensional hypotenuse}

float hypot(float x, float y, float z);

double hypot(double x, double y, double z);

long double hypot(long double x, long double y, long double z);

\vspace{2ex}
\ldots
\vspace{2ex}

\highlight{constexpr} float ceil(float x); \stdcomment{library.c}

\highlight{constexpr} double ceil(double x);

\highlight{constexpr} long double ceil(long double x); \stdcomment{library.c}

\highlight{constexpr} float ceilf(float x);

\highlight{constexpr} long double ceill(long double x);

\vspace{2ex}

\highlight{constexpr} float floor(float x); \stdcomment{library.c}

\highlight{constexpr} double floor(double x);

\highlight{constexpr} long double floor(long double x); \stdcomment{library.c}

\highlight{constexpr} float floorf(float x);

\highlight{constexpr} long double floorl(long double x);

\vspace{2ex}

float nearbyint(float x); \stdcomment{library.c}

double nearbyint(double x);

long double nearbyint(long double x); \stdcomment{library.c}

 float nearbyintf(float x);

 long double nearbyintl(long double x);

\vspace{2ex}

 float rint(float x); \stdcomment{library.c}

 double rint(double x);

long double rint(long double x); \stdcomment{library.c}

 float rintf(float x);

 long double rintl(long double x);

\vspace{2ex}

long int lrint(float x); \stdcomment{library.c}

 long int lrint(double x);

long int lrint(long double x); \stdcomment{library.c}

 long int lrintf(float x);

long int lrintl(long double x);

\vspace{2ex}

long long int llrint(float x); \stdcomment{library.c}

long long int llrint(double x);

long long int llrint(long double x); \stdcomment{library.c}

long long int llrintf(float x);

long long int llrintl(long double x);

\vspace{2ex}

\highlight{constexpr} float round(float x); \stdcomment{library.c}

\highlight{constexpr} double round(double x);

\highlight{constexpr} long double round(long double x); \stdcomment{library.c}

\highlight{constexpr} float roundf(float x);

\highlight{constexpr} long double roundl(long double x);

\vspace{2ex}

\highlight{constexpr} long int lround(float x); \stdcomment{library.c}

\highlight{constexpr} long int lround(double x);

\highlight{constexpr} long int lround(long double x); \stdcomment{library.c}

\highlight{constexpr} long int lroundf(float x);

\highlight{constexpr} long int lroundl(long double x);

\vspace{2ex}

\highlight{constexpr} long long int llround(float x); \stdcomment{library.c}

\highlight{constexpr} long long int llround(double x);

\highlight{constexpr} long long int llround(long double x); \stdcomment{library.c}

\highlight{constexpr} long long int llroundf(float x);

\highlight{constexpr} long long int llroundl(long double x);

\vspace{2ex}

\highlight{constexpr} float trunc(float x); \stdcomment{library.c}

\highlight{constexpr} double trunc(double x);

\highlight{constexpr} long double trunc(long double x); \stdcomment{library.c}

\highlight{constexpr} float truncf(float x);

\highlight{constexpr} long double truncl(long double x);

\vspace{2ex}

\highlight{constexpr} float fmod(float x, float y); \stdcomment{library.c}

\highlight{constexpr} double fmod(double x, double y);

\highlight{constexpr} long double fmod(long double x, long double y); \stdcomment{library.c}

\highlight{constexpr} float fmodf(float x, float y);

\highlight{constexpr} long double fmodl(long double x, long double y);

\vspace{2ex}

\highlight{constexpr} float remainder(float x, float y); \stdcomment{library.c}

\highlight{constexpr} double remainder(double x, double y);

\highlight{constexpr} long double remainder(long double x, long double y); \stdcomment{library.c}

\highlight{constexpr} float remainderf(float x, float y);

\highlight{constexpr} long double remainderl(long double x, long double y);

\vspace{2ex}

float remquo(float x, float y, int* quo); \stdcomment{library.c}

double remquo(double x, double y, int* quo);

long double remquo(long double x, long double y, int* quo); \stdcomment{library.c}

float remquof(float x, float y, int* quo);

long double remquol(long double x, long double y, int* quo);

\vspace{2ex}

\highlight{constexpr} float copysign(float x, float y); \stdcomment{library.c}

\highlight{constexpr} double copysign(double x, double y);

\highlight{constexpr} long double copysign(long double x, long double y); \stdcomment{library.c}

\highlight{constexpr} float copysignf(float x, float y);

\highlight{constexpr} long double copysignl(long double x, long double y);

\vspace{2ex}

double nan(const char* tagp);

float nanf(const char* tagp);

long double nanl(const char* tagp);

\vspace{2ex}

\highlight{constexpr}  float nextafter(float x, float y); \stdcomment{library.c}

\highlight{constexpr}  double nextafter(double x, double y);

\highlight{constexpr}  long double nextafter(long double x, long double y); \stdcomment{library.c}

\highlight{constexpr}  float nextafterf(float x, float y);

\highlight{constexpr}  long double nextafterl(long double x, long double y);

\vspace{2ex}

\highlight{constexpr}  float nexttoward(float x, long double y); \stdcomment{library.c}

\highlight{constexpr}  double nexttoward(double x, long double y);

\highlight{constexpr}  long double nexttoward(long double x, long double y); \stdcomment{library.c}

\highlight{constexpr}  float nexttowardf(float x, long double y);

\highlight{constexpr}  long double nexttowardl(long double x, long double y);

\vspace{2ex}

\highlight{constexpr}  float fdim(float x, float y); \stdcomment{library.c}

\highlight{constexpr}  double fdim(double x, double y);

\highlight{constexpr}  long double fdim(long double x, long double y); \stdcomment{library.c}

\highlight{constexpr}  float fdimf(float x, float y);

\highlight{constexpr}  long double fdiml(long double x, long double y);

\vspace{2ex}

\highlight{constexpr}  float fmax(float x, float y); \stdcomment{library.c}

\highlight{constexpr}  double fmax(double x, double y);

\highlight{constexpr}  long double fmax(long double x, long double y); \stdcomment{library.c}

\highlight{constexpr}  float fmaxf(float x, float y);

\highlight{constexpr}  long double fmaxl(long double x, long double y);

\vspace{2ex}

\highlight{constexpr} float fmin(float x, float y); \stdcomment{library.c}

\highlight{constexpr}  double fmin(double x, double y);

\highlight{constexpr}  long double fmin(long double x, long double y); \stdcomment{library.c}

\highlight{constexpr}  float fminf(float x, float y);

\highlight{constexpr} long double fminl(long double x, long double y);

\vspace{2ex}

\highlight{constexpr}  float fma(float x, float y, float z); \stdcomment{library.c}

\highlight{constexpr}  double fma(double x, double y, double z);

\highlight{constexpr}  long double fma(long double x, long double y, long double z); \stdcomment{library.c}

\highlight{constexpr}  float fmaf(float x, float y, float z);

\highlight{constexpr}  long double fmal(long double x, long double y, long double z);

\vspace{2ex}

// [c.math.fpclass], {\em classification / comparison functions:}

\highlight{constexpr} int fpclassify(float x);

\highlight{constexpr} int fpclassify(double x);

\highlight{constexpr} int fpclassify(long double x);

\vspace{2ex}

\highlight{constexpr} int isfinite(float x);

\highlight{constexpr} int isfinite(double x);

\highlight{constexpr} int isfinite(long double x);

\vspace{2ex}

\highlight{constexpr} int isinf(float x);

\highlight{constexpr} int isinf(double x);

\highlight{constexpr} int isinf(long double x);

\vspace{2ex}

\highlight{constexpr} int isnan(float x);

\highlight{constexpr} int isnan(double x);

\highlight{constexpr} int isnan(long double x);

\vspace{2ex}

\highlight{constexpr} int isnormal(float x);

\highlight{constexpr} int isnormal(double x);

\highlight{constexpr} int isnormal(long double x);

\vspace{2ex}

\highlight{constexpr} int signbit(float x);

\highlight{constexpr} int signbit(double x);

\highlight{constexpr} int signbit(long double x);

\vspace{2ex}

\highlight{constexpr} int isgreater(float x, float y);

\highlight{constexpr} int isgreater(double x, double y);

\highlight{constexpr} int isgreater(long double x, long double y);

\vspace{2ex}

\highlight{constexpr} int isgreaterequal(float x, float y);

\highlight{constexpr} int isgreaterequal(double x, double y);

\highlight{constexpr} int isgreaterequal(long double x, long double y);

\vspace{2ex}

\highlight{constexpr} int isless(float x, float y);

\highlight{constexpr} int isless(double x, double y);

\highlight{constexpr} int isless(long double x, long double y);

\vspace{2ex}

\highlight{constexpr} int islessequal(float x, float y);

\highlight{constexpr} int islessequal(double x, double y);

\highlight{constexpr} int islessequal(long double x, long double y);

\vspace{2ex}

\highlight{constexpr} int islessgreater(float x, float y);

\highlight{constexpr} int islessgreater(double x, double y);

\highlight{constexpr} int islessgreater(long double x, long double y);

\vspace{2ex}

\highlight{constexpr} int isunordered(float x, float y);

\highlight{constexpr} int isunordered(double x, double y);

\highlight{constexpr} int isunordered(long double x, long double y);
}

\subsection{Modifications to  ``Absolute Values''  [c.math.abs]}

\ldots
\vspace{2ex}

\code{
	\highlight{constexpr}  int abs(int j);
	
	\highlight{constexpr} long int abs(long int j);
	
	\highlight{constexpr} long long int abs(long long int j);

	\highlight{constexpr} float abs(float j);
	
	\highlight{constexpr}  double abs(double j);

	\highlight{constexpr} long double abs(long double j);
}


%\end{widetext}

\end{document}
