\documentclass[prd,preprint,amsmath,amssymb,nofootinbib,eqsecnum]{revtex4-1}

\usepackage{xspace} %Dude this will blow your mind!
\newcommand{\constexpr}{\code{constexpr}\xspace}
\newcommand{\code}[1]{{\tt #1}}
\newcommand{\header}[1]{{\tt <#1>}}
\newcommand{\cmath}{\header{cmath}}

\newcommand{\FEINVALID}{{\tt FE\_INVALID}}
\newcommand{\FEDIVBYZERO}{{\tt FE\_DIVBYZERO}}
\newcommand{\FEINEXACT}{{\tt FE\_INEXACT}}
\newcommand{\FEUNDERFLOW}{{\tt FE\_UNDERFLOW}}
\newcommand{\FEOVERFLOW}{{\tt FE\_OVERFLOW}}
\newcommand{\Operators}{\ensuremath{+,-,\times,/}}

\newtheorem{criteria}{Citeria}

\begin{document}


\title{\constexpr for \cmath}
\author{Edward J.~Rosten \& Oliver J.~Rosten}
\date{\today}
\maketitle

\section{Introduction}

We propose simple criteria for selecting functions in \cmath\ which should be
declared \constexpr. Applying this, a list of such functions is drawn up.

\section{Motivation \& Scope}

The introduction of \constexpr has facilitated intuitive compile-time
programming. This paper seeks to rectify the current absence of \constexpr in
\cmath, in order to broaden the range of numeric computations that can be
performed using standard library facilities. In this proposal we are restricting
the functions to those which are in a sense no more complicated than the
currently supported operations (\Operators).
From inspection of \cmath, it may
not be immediately obvious precisely which functions should be declared
\constexpr. To put the application of \constexpr on a rigorous footing, we
propose the following:
\begin{criteria}
	A function in \cmath\ shall be declared \constexpr if:
	\begin{enumerate}
		\item When taken to act on the set of extended rational numbers, the function is closed;
		
		\item The function does not modify any if its arguments which have external visibility.
	\end{enumerate}
\label{cri:rational}	
\end{criteria}

It may seem that these criteria are insufficiently restrictive since various
functions satisfying both may, under certain conditions, set global flags.
Specifically \code{errno} may be set and/or the various floating-point
exception flags, \FEDIVBYZERO, \FEINVALID, \FEOVERFLOW, \FEUNDERFLOW\ and
\FEINEXACT\ may be raised. The latter issue is 
faced by the standard
arithmetic operators but these are nevertheless available for use in constant
expressions. The proposed strategy is to mimic the behaviour of the arithmetic
operators.

Specifically, functions declared \constexpr, \emph{when used in a \constexpr
context}, should give a compiler error if division by zero, domain errors or
overflows occur. When not used in a \constexpr context, the various global
flags should be set as normal. This distinction between these two contexts
implies that any implementation cannot be done as a pure library extension.
Consequently, there will be some burden on compiler vendors. However, the first
criterion above ensures that the functions to be changed are in some sense
simple. Both GCC and clang already support \constexpr within \cmath\ to varying extents. 
GCC declares the majority of functions as \constexpr, including
\code{std::sqrt}, \code{std::exp}, \code{std::lgamma}, etc.
While clang
does not, it does offer some functions as 
builtins and is able to use them to perform compile time computations, constant
propagation and so on. It is therefore hoped that any burden implicit in this
proposal is minimal.

\section{Impact On the Standard}

This proposal impacts the existing headers \cmath\ and \header{cstdlib} such
that the changes do not break existing code and do not degrade performance.
However, it is not a pure library extension. The subtlety arises because of the
requirement that some of the functions affected must behave differently
depending on whether or not they are used in a \constexpr context.

\section{Design Decisions}

There is at least one natural candidate in \header{cmath}, namely $\code{abs}$,
to which \constexpr should be applied. But beyond this, it is desirable to
apply \constexpr throughout \cmath\ in a consistent manner. Ideally, one would
like one or more criteria rooted in mathematics. However, the reality is that
this insufficiently restrictive. Nevertheless, mathematics was our starting
point; as such, any condition on functions to be \constexpr must select the
basic arithmetic operations, $+,-,\times,/$ since these are already \constexpr.

Mathematically, a field is closed under both addition and multiplication.
Numeric types do not form a field; however, since the basic arithmetic
operations are already declared \constexpr, this suggests that it may be
possible to utilize a field which captures enough of the properties of numeric
types in order to be useful in formulating criteria for the application of
\constexpr. The rational numbers are the natural candidate since all valid
values of numeric types are elements of this set. The (extended) rationals close over
$+,-,\times,/$ but functions such as \code{exp}, \code{sqrt}, \code{cos},
\code{sin} and so on are excluded by our first criterion.

The subtlety of global flags being set upon encountering floating-point
exceptions presents a challenge. If all functions which can set such flags are
excluded from the list to tag \constexpr, then the remaining list is rather
sparse. To achieve something more useful suggests expanding the set to include
those functions which are `simple enough'. These considerations lead to the
first condition of criteria~\ref{cri:rational}.
Tables~\ref{tab:26.9.1}--\ref{tab:26.9.4} contain the functions in \cmath\
satisfying this criterion and states whether or not they pass the second
criterion as well. To reduce space, the following convention is observed. The
functions listed in section 26.9 are divided into blocks of closely related
functions such as those shown in \ref{tab:example}. 
\begin{table}
	\begin{tabular}{l}
		\code{int ilogb(float arg)}
	\\
		\code{int ilogb(double arg)}
	\\
		\code{int ilogb(long double arg)}
	\\
		\code{int ilogbf(float arg)}
	\\
		\code{int ilogbl(long double arg)}
	\end{tabular}
\caption{Example of a family of functions which appear as a block in the standard.}
\label{tab:example}
\end{table}
Note that while the first three functions are overloads, the fourth and fifth
have differing names.  When classifying those functions which satisfy the first
criterion, we will give just the first function in each such block, with the
understanding that the others are similar in this regard.

\begin{table}[h]
	\begin{tabular}{lcc}
		Function & Pass & Comment
	\\
	\hline \hline
		\code{float frexp(float value, int* exp)} & No & Modifies argument with external visibility
	\\
	\hline
		\code{int ilogb(float arg)} & Yes & May set global variable
	\\
	\hline
		\code{float logb(float arg)} & Yes & May set global variable
	\\
	\hline
		\code{float ldexp(float x, int exp)} & Yes & May set global variable
	\\
	\hline
		\code{float modf(float value, float* iptr)} & No & Modifies argument with external visibility
	\\
	\hline
		\code {float scalbn(float x, int n)} & Yes & May set global variable
	\\
	\hline
		\code {float scalbln(float x, long int n)} & Yes & May set global variable
	\end{tabular}
\caption{Functions declared in \S 26.9.1 which close on the rationals.}
\label{tab:26.9.1}
\end{table}


\begin{table}[h]
	\begin{tabular}{lc}
		Function & Pass
	\\
	\hline \hline
		\code{int abs(int j)} & Yes
	\\
	\hline
		\code{float fabs(float x)} & Yes
	\end{tabular}
\caption{Functions declared in \S 26.9.2 which close on the rationals.}
\label{tab:26.9.2}
\end{table}

\begin{table}[h]
	\begin{tabular}{lcc}
		Function & Pass & Comment
	\\
	\hline \hline
		\code{float ceil(float x)} &  Yes & Some implementations may set a global variable
	\\
	\hline
		\code{float floor(float x)} & Yes & Some implementations may set a global variable
	\\
	\hline
		\code{float nearbyint(float x)} & Yes &
	\\
	\hline
		\code{float rint(float x)} & Yes & May set a global variable
	\\
	\hline
		\code{long int lrint(float x)} &  Yes & May set a global variable
	\\
	\hline
		\code{long long int llrint(float x)} & Yes & May set a global variable 
	\\
	\hline
		\code{float round(float x)} & Yes & May set a global variable
	\\
	\hline
		\code{float lround(float x)} & Yes & May set a global variable
	\\
	\hline
		\code{float llround(float x)} & Yes & May set a global variable
	\\
	\hline
		\code{float trunc(float x)} &  Yes & May set a global variable
	\\
	\hline
		\code{float fmod(float x, float y)} & Yes & May set a global variable
	\\
	\hline
		\code{float remainder(float x, float y} & Yes & May set a global variable
	\\
	\hline
		\code{float remquo(float x, float y, int* quo)} & No & Modifies argument with external visibility
	\\
	\hline
		\code{float copysign(float x, float y)} & Yes &
	\\
	\hline
		\code{float nextafter(float x, float y)} & Yes & May set a global variable
	\\
	\hline
		\code{float nexttoward(float x, long double y)} & Yes & May set a global variable
	\\
	\hline
		\code{float fdim(float x, float y)} & Yes & May set a global variable
	\\
	\hline
		\code{float fmax(float x, float y)} & Yes &
	\\
	\hline
		\code{float fmin(float x, float y)} & Yes &
	\\
	\hline
		\code{float fma(float x, float y, float z)} & Yes & May set a global variable
	\end{tabular}
\caption{Functions declared in \S 26.9.3 which close on the rationals.}
\label{tab:26.9.3}
\end{table}

\begin{table}[h]
	\begin{tabular}{lcc}
		Function & Pass & Comment
	\\
	\hline \hline
		\code{int fpclassify(float x);} & Yes &
	\\
	\hline
		\code{int isfinite(float x)} & Yes &
	\\
	\hline
		\code{int isinf(float x)} & Yes & \constexpr of dubious utility 
	\\
	\hline
		\code{int isnan(float x)} & Yes & \constexpr of dubious utility 
	\\
	\hline
		\code{int isnormal(float x)} & Yes &
	\\
	\hline
		\code{int signbit(float x)} & Yes &
	\\
	\hline
		\code{int isgreater(float x, float y)} & Yes &
	\\
	\hline
		\code{int isgreaterequal(float x, float y)} & Yes &
	\\
	\hline
		\code{int isless(float x, float y)} & Yes &
	\\
	\hline
		\code{int islessequal(float x, float y)} & Yes &
	\\
	\hline
		\code{int islessgreater(float x, float y)} & Yes &
	\\
	\hline
		\code{int isunordered(float x, float y)} & Yes &
	\end{tabular}
\caption{Functions declared in \S 26.9.4 which close on the rationals.}
\label{tab:26.9.4}
\end{table}

\section{Technical Specifications}

% TO DO

\end{document}
