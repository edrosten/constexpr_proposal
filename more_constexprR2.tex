\documentclass[prd,twocolumn,amsmath,amssymb,nofootinbib,eqsecnum]{revtex4-1}
\it
\usepackage{xspace} %Dude this will blow your mind!
\usepackage{amsthm}
\usepackage{color}
\usepackage{soul}
\usepackage{fancyvrb}
\usepackage{epsfig}
\usepackage{hyperref}
\usepackage[normalem]{ulem}
\newcommand{\constexpr}{\code{constexpr}\xspace}
\newcommand{\code}[1]{{\tt #1}}
\newcommand{\header}[1]{{\tt <#1>}}
\newcommand{\cmath}{\header{cmath}}
\newcommand{\complex}{\header{complex}}
\newcommand{\cstdlib}{\header{cstdlib}}

\newcommand{\ie}{i.e.}

\newcommand{\FEINVALID}{{\tt FE\_INVALID}}
\newcommand{\FEDIVBYZERO}{{\tt FE\_DIVBYZERO}}
\newcommand{\FEINEXACT}{{\tt FE\_INEXACT}}
\newcommand{\FEUNDERFLOW}{{\tt FE\_UNDERFLOW}}
\newcommand{\FEOVERFLOW}{{\tt FE\_OVERFLOW}}
\newcommand{\FLTRADIX}{{\tt FLT\_RADIX}}
\newcommand{\Operators}{\ensuremath{+,-,\times,/}}

\newcommand{\highlight}[1]{\colorbox{green}{\!\!\!\! #1}}
\newcommand{\oldhighlight}[1]{#1}
\newcommand{\stdcomment}[1]{{// {\it see} [#1]}}
\newcommand{\remove}[1]{\colorbox{red}{\!\!\!\! #1}}

\newcommand{\eq}[1]{(\ref{eq:#1})}
\newcommand{\fig}[1]{figure~\ref{fig:#1}}
\newcommand{\sect}[1]{section~\ref{sec:#1}}

\newcommand{\bigund}{{\Huge{\_}}}
\newcommand{\fptype}{{\it floating-point-type}}

\newtheorem*{proposal*}{Proposal}

\makeatletter
\def\l@subsection#1#2{}
\def\l@subsubsection#1#2{}
\makeatother

\def\bibsection{\section*{\refname}} 

\begin{document}


\title{More \constexpr for \cmath\ and \complex}
%\author{Edward J.~Rosten \& Oliver J.~Rosten}
%\date{\today}

\author{
\begin{tabular}{ll}
	Document: & P1383R2
\\
	Date: & \today
\\
	Project: & Programming Language C++, Library Working Group
\\
	Audience: & SG6 $\rightarrow$ LEWG $\rightarrow$ LWG
\\
	Reply to: & Oliver J.\ Rosten {(\tt oliver.rosten@gmail.com)}
\end{tabular}
}

\begin{abstract}

\begin{center} {\bf Abstract} \end{center}

A scattering of $\constexpr$, principally throughout \cmath, was proposed in~\cite{Rosten-constexpr} and accepted into C++23. This was subject to a constraint that the affected functions be limited to those which are, in a well-defined sense, no more complicated than the arithmetic operators \Operators. It is proposed to remove this restriction, thereby allowing a richer spectrum of mathematical functions to be used in a \constexpr\ context. 

%The focus of this proposal is to build upon existing practice, with the goal of providing something immediately useful to a section of the C++ user base. However, it is recognized that the mathematical functions considered do not serve those requiring strict platform-independence. It is recognized that this is an important need, but argued that it is orthogonal and therefore befits a separate proposal.

\end{abstract}


\maketitle
\tableofcontents

\section{Revision History}

\begin{enumerate}
	\item[R1] Greatly expanded analysis of the design space.
	
	\item[R2] Rebase wording to N4944 and include overlooked subsections of \complex.
\end{enumerate}

\section{Introduction}
\label{sec:Intro}

Since its inception, \constexpr\ has become an invaluable ingredient in compile-time programming. Indeed, part of its appeal is that the sharp distinction between meta-programming and runtime programming has in many instances become blurred. The interest in \constexpr\ is reflected by the numerous papers proposing to increase the range of core language features and library functionality that may be used in a \constexpr\ context. As such, it is essential for the long-term uniformity of C++ that parts of the standard library are not left behind in this process.

This paper is the natural extension of~\cite{Rosten-constexpr} and seeks to significantly expand the number of functions in \cmath\ (and also \complex) which may be used in a \constexpr context. The potential utility of this for numerics is noteworthy. However, it is clear that there are new hurdles to overcome: floating-point is very subtle and different people may want implementations to prioritize different things~\cite{Timmons-Less}. In other words, there is a non-trivial design space. 

Consider some function $f(x)$ and its floating-point implementation $F_R(x_n)$, where the $x_n$ are numbers representable by the floating-point type being used and $R$ denotes the rounding mode. A central issue is whether, for a given $x_n$ and $R$, $F_R(x_n)$ should always give the same answer, regardless of compiler settings and/or platform. In some cases, the answer is clearly yes: for example if $f$ corresponds to multiplication by two.  However, to take a different extreme, what about \code{std::sin(1e100)}?%
\footnote{Thanks to Richard Smith for first bringing this example to our attention.}

Whilst it is of course true that there is an unambiguous result to $\sin 10^{100}$, we may well wonder how meaningful it is. First, shifting the argument by a tiny amount---of relative size $10^{-100}$---can cause the output to change dramatically. Secondly, the distance between adjacent (double precision) floating-point numbers at this scale is of order $10^{84}$. To put this into perspective, consider that the ratio between the size of the observable universe and the Planck scale%
\footnote{The unimaginably small distance scale at which quantum gravity presumably reigns.}
is `only' of order $10^{62}$. Suppose that we tell an astronomer that we expect them to measure this ratio in order that we can take its $\sin$, and that we further expect to at least get the first few significant figures right. This is not entirely different from expecting a standard library implementer to ensure that \code{std::sin} gives the `correct' answer to the last bit, no matter how large the argument: it is not obvious how to ascribe meaning to the answer.

But how to reconcile this with the other extreme exemplified by the requirement that multiplication by two should always gives the correct answer? Floating-point numbers can be thought of in two ways: as points on the number line or as intervals (c.f.\ interval arithmetic~\cite{Interval}). For a given $x_n$ and rounding mode, the corresponding interval is such that all numbers in the interval collapse to $x_n$ under rounding. The granularity of floating-point is such that $\sin x$ sweeps across its entire range from [-1,+1] an enormous number of times in the interval containing $10^{100}$. Contrariwise, multiplication by two maps the interval associated with $x_n$ into the interval associated with $2x_n$.

Thus, within the interval approach, it is reasonable for different implementations to produce different
values for the same input; or even for the same implementation to do so depending on things such as the degree of optimization. This is true of the way C++ works at present. For an example, see \href{https://godbolt.org/z/M3fhYhx84}{https://godbolt.org/z/M3fhYhx84}:
\begin{verbatim}
float num() { return 3.14f; }

int main()
{
    std::cout << std::hexfloat;
    std::cout << num() / 3.14f;
}
\end{verbatim}
With the compiler setting \code{-O2 -ffast-math} the answer is not precisely one; however, removing the \code{-ffast-math} flag and/or using \code{-O0} it is one. From an interval perspective, this can be rationalized. Perhaps some will baulk at this example due to the use of \code{-ffast-math}; after all it does not conform to IEEE. But sometimes the flexibility afforded by \code{-ffast-math}, such as re-ordering of operations, allows it to give what is, intuitively at any rate, a more mathematically correct answer \href{https://godbolt.org/z/3vehd87h8}{https://godbolt.org/z/3vehd87h8}.
\begin{verbatim}
float big() { return 1e20f;}

int main()
{
    std::cout << big() + 3.14f - big();
}
\end{verbatim}
Removing the \code{-ffast-math} flag causes the output to change from \code{3.14f} to zero. However care needs to be taken over-interpreting this particular example. After all, what is meant by a number such as \code{1e20f}? A set of numbers within a range of order $10^{13}$ reduce to this under rounding. It is simply not expressible within the language which number within this range is meant by \code{1e20f}; and how could it be? Floating-point has finite precision. Within the interval approach, all numbers within the appropriate range (once rounding is taking into account) are equally valid. Following this logic through, from an interval approach, the results from both using and not using \code{-ffast-math} can be considered equally valid. Then again, the interval approach actually permits any answer within what intuitively feels like a large range (though in relative terms, with the scale set by \code{big()}, the range is small).

In more general terms, the interval approach can be understood as follows. Under the action of the true mathematical function, $f$, the mapping of the floating-point interval containing $x$ intersects a union of floating-point intervals. The latter define what can reasonably be considered equally valid output of $F_R(x_n)$. Do the various C++ library implementations conform to this? Doubtful! As far as we are 
aware implementations were not developed with this in mind.
Attempting to remedy this via standardization seems
counterproductive. The effect would likely be to render all the existing implementations---which for many purposes work perfectly well---non-conforming. Rather, it is suggested that consideration of intervals can,
in principle, be used as one measure of the Quality of Implementation (QoI).

With all this in mind, the design space divides up as follows:
\begin{enumerate}
	\item[Canon] The output of the floating-point implementation of a mathematical function coincides 
	with that of the underlying function itself, down to the last bit.
	
	\item[Interval] The output of the floating-point implementation of a mathematical function may fall 
	within a range, generated by considering the mapping of intervals corresponding to floating-point 
	numbers.
	
	\item[Approx] A weaker version of [Interval], where deviations from either the desired output range 
	or the canonical answer may be taken to reflect the QoI.
\end{enumerate}

We assume that C++ implementations currently fall within the last of these (if any fall within the second that would be wonderful, but there is a burden of proof to demonstrate this to be the case). Is this a problem? It depends. If you are writing a networked multiplayer game, then lack of support for [Canon] may be an issue. On the other hand, if you are doing physical modelling where the errors in the boundary conditions are large compared to the floating-point granularity at the appropriate scale then [Approx] may be perfectly sound. Indeed, [Approx] may allow vendors to write faster implementations than [Canon] and why should users pay for a spurious `improvement' in accuracy?

All of this suggests the following avenues (non-exhaustive and not all of which are mutually-exclusive) for standardization.
\begin{enumerate}
	\item Allow existing [Approx] implementations of the mathematical functions to be used in a \constexpr\ context demanding either
	\begin{enumerate}
		\item No additional requirements, whatsoever;
		
		\item No hard requirements but encouraging QoI to be specified in terms of degree
		of adherence to either [Canon] or [Interval];
		\label{approx-with-QoI}
		
		\item That, in a \constexpr\ context, functions satisfy [Canon];
		\label{canon-constexpr}
		
		\item That, in a \constexpr\ context, functions satisfy [Interval].
	\end{enumerate}

	\item Introduce a new set of mathematical functions, satisfying the requirements of [Canon], and
	allow them to be used in a \constexpr\ context.
	\label{canon-new}
	
	\item Demand existing functions always satisfy the constraints of [Interval].
	
	\item Demand existing functions be [Canon] and introduce a new set with weaker constraints.
\end{enumerate}
Actually, as far as this paper is concerned the final option is not viable. First and foremost, this would compel compiler vendors to change the behaviour of existing implementations, which could have unexpected effects on the behaviour and performance of programs. Beyond this, as alluded to above, it is not obvious whether it is constructive to demand [Canon] in all circumstances. Insisting on the `correct' answer to the last bit can be an exercise in absurdity. Perhaps it may be reasonable to prescribe certain answers---e.g.\ zero for \code{std::sin(1e100)}---but how to agree on this value and how to decide where evaluation switches from computation to prescription?%
\footnote{One reasonable approach would be to make the shift when the floating-point interval between numbers becomes sufficiently large so as to saturate the range of a function.
}
 There is also the question as to whether relationships such as the trigonometric identities should be preserved%
\footnote{Thanks to Hans Boehm for pointing this out.}; but again it is worth at least critically assessing whether floating-point calculations have any real meaning at all in certain domains. Finally, even away from awkward cases, actually verifying that answers are canonical could be difficult, especially for \code{double}s.

As for the third option, this has a certain appeal. However, on balance it is preferred to encourage implementors to use this (or something along similar lines) to quantify their QoI. Insisting that existing implementations strictly adhere to this seems too prescriptive.

While there may be merit in the second option above, this paper proposes standardization of~\ref{approx-with-QoI}. This does not preclude proposing the second option at a future date, though a considered resolution of how to deal with things like \code{std::sin(1e100)} should be found. 
Any potential tension is more about the philosophical question of whether [Approx] functions should be usable in a \constexpr\ context. Strict adherents to the position of only allowing [Canon] functions to be used in this way may object to the first option out of principle but, again, let us emphasise that practically speaking there is no reason the two approaches cannot coexists.

To conclude the introduction, given our belief that declaring more functions within \cmath\ to be \constexpr\ is useful, we seek to adhere to one of the core principles of C++~\cite{DandE}:
\begin{quote}
	It is more important to allow a useful feature than to prevent every misuse.
\end{quote}
Whether or not proposal~\ref{approx-with-QoI} is ultimately accepted boils down to the relative strength of usefulness and implementability versus potential for misuse. For~\ref{canon-constexpr} the balancing act shifts more towards whether the impact on library implementors and the commensurate delay for end-users is worth it: doing this is as hard as~\ref{canon-new}. It is also worth noting that the~\ref{canon-constexpr} approach of  switching between [Appox] and [Canon] implementations via an \code{if consteval} branch will, in some cases, significantly increase the difference between the compile-time and runtime outputs, compared to~\ref{approx-with-QoI}.

\section{Motivation \& Scope}
\label{sec:Motivation}

Prior to~\cite{Rosten-constexpr}, no effort had been made to allow for functions in \cmath\ to be declared \constexpr; this despite there being glaring instances, such as \code{std::abs}, for which this was arguably perverse. Indeed, between~\cite{AP-complex} and~\cite{Rosten-constexpr} being adopted, the situation was actually been better for \complex\ than for \cmath! The aim of~\cite{Rosten-constexpr} was to at least partially rectify the situation, while recognizing that attempting to completely resolve this issue in a single shot was too ambitious.

The broad strategy of~\cite{Rosten-constexpr} is to focus on those functions which are, in a well-defined sense, no more complicated than the arithmetic operators \Operators; the rationale for this being that the latter are already available in a \constexpr\ context.
As~\cite{Rosten-constexpr} proceeded through the standardisation process, LEWG expressed a desire 
to extend the scope to include a significant amount of what remains in \cmath, in particular common mathematical functions such as \code{std::exp}.%
\footnote{It seems too ambitious at this stage to include the mathematical special functions [sf.cmath] and so they are excluded from this proposal.}
However, later discussion---crystalized in~\cite{Timmons-Less}---revealed significant worries.

Various, related concerns to the goal of this paper to declare more functions in \cmath\ to be \constexpr\ have been raised:
\begin{enumerate}
	\item Implementations of certain functions in \cmath\ do not produce results which are `correctly' rounded to the last bit.%
	\footnote{As discussed in the introduction, it is doubtful whether rounding of e.g.\ \code{std:sin(1e100)} can be meaningfully considered correct, though it may nevertheless serve as a useful prescription for giving a standard answer.}	
			
	\item The output of certain functions in \cmath\ may differ depending on whether they are evaluated at runtime or translation time.
	
	\item The output of certain functions in \cmath\ may vary from platform to platform.
	
	\item The output of certain functions in \cmath\ may depend on the level of optimization; a corollary
	is that even with a given level of (non-trivial) optimization, the same function in \cmath\ may give 
	different answers, depending on the ambient 
	code \href{https://godbolt.org/z/js7rGvPbf}{https://godbolt.org/z/js7rGvPbf}.%
	\footnote{Thanks to Matthias Kretz for supplying this example.}
	
	\item The output of certain functions in \cmath\ may differ when the same binary is executed on
	different CPUs within the same architectural family.

\end{enumerate}

Actually, concerns like these can be subsumed into three broader worries:
\begin{enumerate}
	\item Is it acceptable for evaluation of mathematical functions to differ between translation time and
	runtime?
	
	\item Is is acceptable for the evaluation of mathematical functions at runtime to depend on things other than the rounding mode?

	\item Is it acceptable for constant evaluation of mathematical functions to differ between platforms?
\end{enumerate}

A conceptual framework for reasoning about these issues has been given in the introduction. If one had the luxury of possessing both a [Canon] and [Interval] implementation then much of the concern regarding the first and second issues could be allayed by simply choosing the appropriate implementation for the task in hand, accepting the tradeoffs. If an identical answer is required in all situations (except possibly when the runtime rounding mode is changed), then use the [Canon] implementation. If the cost of this is undesirable then use [Interval]: the same function may produce different results in different contexts, but in a manner such that all outputs are, in a well defined sense, equally valid.

However, the reality is that the only available implementations are [Approx]. Nevertheless, C++ has long allowed for differences between translation time and runtime; and runtime evaluation itself depends on things such as the degree of optimization. It is certainly the case that quantification of this may very well be useful and can feed into QoI; but in of itself long-standing practice shouldn't present an obstacle to the goals of this paper. But what of whether it is acceptable for constant evaluation of mathematical functions to differ between platforms? Even if an [Interval] implementation were available, some disquiet may follow from the fact that different platforms may do different things in the following example:
\begin{verbatim}
template<double D>
struct do_stuff
{
  static void execute() {}
};

template<>
struct do_stuff<1.0>
{
  static void execute() { destroy_everything(); }
};

// Do I feel lucky?
do_stuff<std::sin(1e100)>::execute();
\end{verbatim}

That being said, even if the result were guaranteed to be the same on every platform, the code is no less ridiculous. Indeed, we believe that the utility of rolling out \constexpr\ to touch more of \cmath\ outweighs the fact that it may be misused, bringing us back to the core principle cited in the introduction, from~\cite{DandE}.
It is perfectly reasonable for people to want to generate a \constexpr\ lookup table for (say) \code{std::sin}. If they are operating in the domain where they do not care whether their values differ between platforms etc.\ is it really right that we continue to prohibit this entirely legitimate use-case?

Furthermore, there is a case to be made that allowing additional functions within \cmath\ to be used in constant expressions actually confers a greater degree of control to users. As things stand, whether constant folding is performed can depend on a variety of factors, such as the degree of optimization and the ambient code (as illustrated earlier). By allowing people to write things such as
\begin{verbatim}
constexpr auto x{std::exp(1.0)};
\end{verbatim}
 they are able to specify their intent within the language: evaluate this at translation time.

This naturally leads us to consider the fact that the latest C working draft carves out identifiers prefixed with \code{cr\_} to represent [Canon] implementations of mathematical functions. If we are to work on the assumption that such functions will at some point materialize, then why not leave \cmath\ as it is and only allow the \code{cr\_} functions in a \constexpr\ context? After all, in the above example the intent to evaluate at translation time could be perfectly well specified, without any appreciable downside, by
\begin{verbatim}
constexpr auto x{std::cr_exp(1.0)};
\end{verbatim}
But here's the problem: mathematical functions are used to build up other functions and it may be desirable to use such functions both in a standard runtime context and also in a \constexpr\ context, for all the reasons given above. For users who do not care that these functions are not [Canon]---indeed, they may very well desire the runtime performance gains that a less prescriptive implementation may confer---it is not reasonable to expect them to implement all of their functions twice.

Thus, regardless of whether or not C++ ends up with [Canon] mathematical functions
\begin{enumerate}
	\item There will always be a desire from some quarters for the existing \cmath\ behaviour;
	
	\item Allowing people to use these functions in a \constexpr\ context has benefits.
\end{enumerate}


\section{State of the Art and Impact on Implementers}

\subsection{Current Implementations}

With the exception of the special functions [sf.cmath], functions taking a pointer argument and those with an explicit dependence on the runtime rounding mode, GCC currently renders almost everything in \cmath\ \constexpr. Though clang does not have \constexpr implementations, it does perform compile time evaluation of many mathematical functions (but not the special functions) during optimization. The existence of compile time evaluation in GCC and clang demonstrates that implementation of this proposal is plausibly feasible. 

Nevertheless, even for GCC's implementation of the relatively simple functions which~\cite{Rosten-constexpr} declares \constexpr, there are subtleties. In particular GCC is not entirely consistent with the way in which it presently deals with NaNs and/or infinities when they are passed as arguments to various mathematical functions.


\subsection{Special Values}

Two problems that~\cite{Rosten-constexpr} had to deal with was situations in which
\begin{enumerate}

\item Floating-point exceptions (other than \FEINEXACT) are raised;

\item NaNs and/or infinities are passed as arguments to functions in \cmath\ declared \constexpr.

\end{enumerate}
The chosen solution was to delegate to Annex F of the C standard insofar as it applicable.
Recall that Annex F specifies C language support for IEC 60559 arithmetic; thus, to the extent that a
floating-point type conforms to this, the behaviour in the aforementioned situations is 
exactly prescribed in C++, following the adoption of~\cite{Rosten-constexpr}. Should a
floating-point type not conform to relevant parts of IEC 60559, then its behaviour in these
situations is unspecified.

This strategy is applicable in its current form to this paper, though the range of scenarios
in which Annex F may be invoked is somewhat richer. For example, an implementation
conforming to IEC 60559 must give \code{acos(1) = +0}.

\subsection{Interaction with the C Standard Library}

For a mathematical function which may be evaluated at translation time, 
putting all peculiarities of floating-point momentarily to one side, it is desirable for there to be
consistency with the values computed at runtime. However, the fact that the rounding mode
may be changed at runtime indicates that this is not, in general, possible. 

For more complicated mathematical functions there is an additional subtlety due
to the interaction with the C standard library. In [library.c] it is noted that \cmath\ makes available the facilities of the C standard library. One interpretation of this is that the C++
implementation could use one of several different C standard libraries. If so, 
constraining translation time behaviour so that it is consistent with the runtime behaviour
could be very difficult, quite apart from the issue of the runtime rounding mode.

Let us return to an earlier example:
\begin{Verbatim}
#include <cmath>
double f() { return std::sin(1e100); }
\end{Verbatim}
It turns out that on clang (targeting x64), the following code is emitted:
\begin{Verbatim}
.LCPI0_0:
	.quad	-4622843457162800295 
_Z1fv:
	movsd	.LCPI0_0(%rip), %xmm0
	retq
\end{Verbatim}
with equivalent code generated by GCC. This demonstrates that both compilers are already generating the results at translation time and, therefore, independently of the runtime C library. For this particular example, it appears that current practise does indeed achieve consistency between translation time and runtime, though effectively by ignoring the latter!

The story does not end here. For more complicated examples and/or removing optimization, it may be that a runtime call to the C library is made, after all. Bearing in mind that any value in the range [-1, 1] could be considered reasonable, this implies that the value of, say, \code{std::sin(1e100)} evaluated in one part of a code base may be very different from the (translation time) value evaluated elsewhere. 
Nevertheless, it seems reasonable in our opinion that both clang and GCC tacitly allow this, as already discussed in detail.



\section{Design Decisions}

The key design decisions advocated in this paper are that:
\begin{enumerate}
	\item It is acceptable for evaluation of mathematical functions to differ between translation time and runtime.

	\item It is acceptable for constant evaluation of mathematical functions to differ between platforms.
	
	\item It is preferable to encourage quantification of QoI rather than mandate precise behaviour for existing functions within \cmath.
\end{enumerate}

Let us recapitulate the various points.

\begin{enumerate}
	\item Allowing a broader range of mathematical functions to be used within constant expressions is useful; GCC already supports this.
	
	\item Since the advent of \constexpr, the standard has implicitly allowed for differences between translation time and runtime evaluation: the arithmetic operators \Operators\ may be used in either context, but only in a runtime context may the rounding mode be changed. Furthermore, runtime evaluation may invoke instructions such as fused multiply-add (fma), which are not necessarily utilized at translation time: \href{https://godbolt.org/z/ceonfG4cT}{https://godbolt.org/z/ceonfG4cT}.
	
	\item Even without \constexpr, current practice has long allowed for differences in the output of mathematical functions between any of translation time, runtime, runtime with different compiler flags, and runtime on a different platform. For example, optimization may emit code which entirely bypasses runtime calls to the C library, instead generating results at translation time. However, under other circumstances, optimization might not do this.
	
	\item The philosophy of this paper is not to accept an impasse. Rather, it is preferred to support
	a useful extension to existing practice in a non-prejudicial fashion, while not precluding 
	orthogonal developments which may cater for a different range of use-cases. It also advocates
	(but does not require) that implementers provide more information on their QoI.
\end{enumerate}



One way or another, much of this boils down to the question of whether it is really acceptable for mathematical functions to give different results in different contexts, given the same input. Again, our answer is yes; we emphasise again that this is already part of C++ and does not imply that C++ is broken! That being said, there are cases where people may want a mathematical function to produce the same result, given the same input, in all situations (except, presumably, when the rounding mode is changed). Our opinion is that this is best served by a separate proposal.

\section{Impact On the Standard}

This proposal amounts to a (further) liberal sprinkling of \constexpr\ in \cmath, together with a smattering in \complex.

\section{Future Directions}

It is worth considering separate implementations of mathematical functions with strict guarantees on their outputs. Finally, it would be ultimately desirable to extend \constexpr\ to some, if not all, of the special functions. 


\begin{acknowledgments}
	I would like to thank Richard Smith for his usual perceptive comments and Matthias Kretz and Nick Timmons for some very helpful feedback and discussions. Particular thanks to Ed Rosten for collaborating on the first incarnation of this paper.
\end{acknowledgments}


\begin{thebibliography}{1}
	\bibitem[P0533]{Rosten-constexpr} Edward J.~Rosten and Oliver J.~Rosten, \constexpr\ for \cmath\ and \cstdlib.

	\bibitem[P2337]{Timmons-Less} Nicholas G. Timmons, Less \constexpr\ for \cmath.
	
	\bibitem[D\&E]{DandE} Bjarne Stroustrup, The Design and Evolution of C++.

	\bibitem[P0415R0]{AP-complex} Antony Polukhin, Constexpr for std::complex.	
		
	\bibitem[N4944]{WorkingPaper} Thomas K\"oppe, ed., Working Draft, Standard for Programming Language C++.	
	
	\bibitem[Kahan]{Interval} William Kahan. 2006. How Futile Are Mindless Assessments of Roundoff in Floating-Point Computation? Retrieved \today\ from \url{https://people.eecs.berkeley.edu/?wkahan/Mindless.pdf}.
	
	%\bibitem[Rudin 1976]{BabyRudin} Walter Rudin, Principles of Mathematical Analysis, 3rd Edition.
\end{thebibliography}

\newpage

%\begin{widetext}
\onecolumngrid

\section{Proposed Wording}

\setlength{\parindent}{0pt}


The following proposed changes, indicated in \highlight{green}, refer to the Working Paper~\cite{WorkingPaper}


\subsection{Modifications to ``Header \header{complex} synposis'' [complex.syn]}

// [complex.value.ops], {\it values }

\vspace{2ex}

\code{
  	template<class T> constexpr T real(const complex<T>\&);
	
  	template<class T> constexpr T imag(const complex<T>\&);

	\vspace{2ex}

	template<class T> \highlight{constexpr} T abs(const complex<T>\&);
 	
	template<class T> \highlight{constexpr} T arg(const complex<T>\&);
	
	template<class T> constexpr T norm(const complex<T>\&);
	
	\vspace{2ex}
	
	template<class T> constexpr complex<T> conj(const complex<T>\&);
	
	template<class T> \highlight{constexpr} complex<T> proj(const complex<T>\&);
	
	template<class T> \highlight{constexpr} complex<T>  polar(const T\&, const T\& = T());	

}

\vspace{2ex}

// [complex.transcendentals], {\it transcendentals}

\code{

\vspace{2ex}
  template<class T> \highlight{constexpr} complex<T> acos(const complex<T>\&);

  template<class T> \highlight{constexpr} complex<T> asin(const complex<T>\&);

  template<class T> \highlight{constexpr} complex<T> atan(const complex<T>\&);

\vspace{2ex}

  template<class T> \highlight{constexpr} complex<T> acosh(const complex<T>\&);

  template<class T> \highlight{constexpr} complex<T> asinh(const complex<T>\&);

  template<class T> \highlight{constexpr} complex<T> atanh(const complex<T>\&);

\vspace{2ex}

  template<class T> \highlight{constexpr} complex<T> cos  (const complex<T>\&);

  template<class T> \highlight{constexpr} complex<T> cosh (const complex<T>\&);

  template<class T> \highlight{constexpr} complex<T> exp  (const complex<T>\&);

  template<class T> \highlight{constexpr} complex<T> log  (const complex<T>\&);

  template<class T> \highlight{constexpr} complex<T> log10(const complex<T>\&);

\vspace{2ex}

  template<class T> \highlight{constexpr} complex<T> pow  (const complex<T>\&, const T\&);

  template<class T> \highlight{constexpr} complex<T> pow  (const complex<T>\&, const complex<T>\&);

  template<class T> \highlight{constexpr} complex<T> pow  (const T\&, const complex<T>\&);

\vspace{2ex}

  template<class T> \highlight{constexpr} complex<T> sin  (const complex<T>\&);

  template<class T> \highlight{constexpr} complex<T> sinh (const complex<T>\&);

  template<class T> \highlight{constexpr} complex<T> sqrt (const complex<T>\&);

  template<class T> \highlight{constexpr} complex<T> tan  (const complex<T>\&);

  template<class T> \highlight{constexpr} complex<T> tanh (const complex<T>\&);
}

\subsection{Modifications to ``Value Operations'' [complex.value.ops]}

\code{
	\ldots\\

	template<class T> \highlight{constexpr} T abs(const complex<T>\& x);\\

 	\ldots\\

	template<class T> \highlight{constexpr} T arg(const complex<T>\& x);\\

	\ldots\\

	template<class T> constexpr T norm(const complex<T>\& x);\\

	\ldots\\

	template<class T> constexpr complex<T> conj(const complex<T>\& x);\\

	\ldots\\

	template<class T> \highlight{constexpr} complex<T> proj(const complex<T>\& x);\\

	\ldots\\

	template<class T> \highlight{constexpr} complex<T> polar(const T\& rho, const T\& theta = T());	
}

\subsection{Modifications to ``Transcendentals'' [complex.transcendentals]}

\code{
  	template<class T> \highlight{constexpr} complex<T> acos(const complex<T>\& x);\\

	\ldots\\

	template<class T> \highlight{constexpr} complex<T> asin(const complex<T>\& x);\\

	\ldots\\

	template<class T> \highlight{constexpr} complex<T> atan(const complex<T>\& x);\\

	\ldots\\

  	template<class T> \highlight{constexpr} complex<T> acosh(const complex<T>\& x);\\

	\ldots\\

	template<class T> \highlight{constexpr} complex<T> asinh(const complex<T>\& x);\\

	\ldots\\

	template<class T> \highlight{constexpr} complex<T> atanh(const complex<T>\& x);\\

	\ldots\\

	template<class T> \highlight{constexpr} complex<T> cos  (const complex<T>\& x);\\

	\ldots\\

	template<class T> \highlight{constexpr} complex<T> cosh (const complex<T>\& x);\\

	\ldots\\
	
	template<class T> \highlight{constexpr} complex<T> exp  (const complex<T>\& x);\\

	\ldots\\

 	template<class T> \highlight{constexpr} complex<T> log  (const complex<T>\& x);\\

	\ldots\\

	template<class T> \highlight{constexpr} complex<T> log10(const complex<T>\& x);\\

	\ldots\\
 
 	template<class T> \highlight{constexpr} complex<T> pow  (const complex<T>\& x, const complex<T>\& y);\\	
	\ldots\\

  	template<class T> \highlight{constexpr} complex<T> pow  (const complex<T>\& x, const T\& y);\\
	
	\ldots\\

	template<class T> \highlight{constexpr} complex<T> pow  (const T\& x, const complex<T>\& y);\\

	\ldots\\

  	template<class T> \highlight{constexpr} complex<T> sin  (const complex<T>\& x);\\

	\ldots\\
	
	template<class T> \highlight{constexpr} complex<T> sinh (const complex<T>\& x);\\

	\ldots\\
 
 	template<class T> \highlight{constexpr} complex<T> sqrt (const complex<T>\& x);\\

	\ldots\\
 
 	template<class T> \highlight{constexpr} complex<T> tan  (const complex<T>\& x);\\

	\ldots\\

	template<class T> \highlight{constexpr} complex<T> tanh (const complex<T>\& x);\\

	\ldots\\
}

\subsection{Modifications to  ``Additional overloads'' [cmplx.over]}

$^1$ The following function templates \remove{shall} have additional \highlight{constexpr} overloads:
\newline
\hspace{5em}
\code{
\begin{tabular}{lll}
	arg &\qquad \qquad\qquad&norm
\\
	conj & \qquad\qquad\qquad&proj
\\ 
	imag & \qquad \qquad\qquad&real
\end{tabular}
}
\newline
\remove{where \code{norm}, \code{conj}, \code{imag}, and \code{real} are constexpr overloads.}

$^2$ The additional \highlight{constexpr} overloads \remove{shall be} \highlight{are} sufficient to ensure:
\\
\ldots

$^3$ Function template \code{pow} has additional \highlight{constexpr} overloads sufficient to ensure\ldots


\subsection{Modifications to  ``Header \header{cmath} synopsis'' [cmath.syn]}


\code{

namespace std\{

\vspace{2ex}
\ldots
\vspace{2ex}

%% acos %%

\highlight{constexpr} \fptype\ acos(\fptype\ x);

\highlight{constexpr} float acosf(float x);

\highlight{constexpr} long double acosl(long double x);

\vspace{2ex}

%% asin %%

\highlight{constexpr} \fptype\ asin(\fptype\ x);

\highlight{constexpr} float asinf(float x);

\highlight{constexpr} long double asinl(long double x);

\vspace{2ex}

%% atan %%

\highlight{constexpr} \fptype\ atan(\fptype\ x);

\highlight{constexpr} float atanf(float x);

\highlight{constexpr} long double atanl(long double x);

\vspace{2ex}

%% atan2 %%

\highlight{constexpr} \fptype\ atan2(\fptype\ y, \fptype\ x);

\highlight{constexpr} float atan2f(float y, float x);

\highlight{constexpr} long double atan2l(long double y, long double x);

\vspace{2ex}

%% cos %%

\highlight{constexpr} \fptype\ cos(\fptype\ x);

\highlight{constexpr} float cosf(float x);

\highlight{constexpr} long double cosl(long double x);

\vspace{2ex}

%% sin %%

\highlight{constexpr} \fptype\ sin(\fptype\ x);

\highlight{constexpr} float sinf(float x);

\highlight{constexpr} long double sinl(long double x);

\vspace{2ex}

%% tan %%

\highlight{constexpr} \fptype\ tan(\fptype\ x);

\highlight{constexpr} float tanf(float x);

\highlight{constexpr} long double tanl(long double x);

\vspace{2ex}

%% acosh %%

\highlight{constexpr} \fptype\ acosh(\fptype\ x);

\highlight{constexpr} float acoshf(float x);

\highlight{constexpr} long double acoshl(long double x);

\vspace{2ex}

%% asinh %%

\highlight{constexpr} \fptype\ asinh(\fptype\ x);

\highlight{constexpr} float asinhf(float x);

\highlight{constexpr} long double asinhl(long double x);

\vspace{2ex}

%% atanh %%

\highlight{constexpr} \fptype\ atanh(\fptype\ x);

\highlight{constexpr} float atanhf(float x);

\highlight{constexpr} long double atanhl(long double x);

\vspace{2ex}

%% cosh %%

\highlight{constexpr} \fptype\ cosh(\fptype\ x);

\highlight{constexpr} float coshf(float x);

\highlight{constexpr} long double coshl(long double x);

\vspace{2ex}

%% sinh %%

\highlight{constexpr} \fptype\ sinh(\fptype\ x);

\highlight{constexpr} float sinhf(float x);

\highlight{constexpr} long double sinhl(long double x);

\vspace{2ex}

%% tanh %%

\highlight{constexpr} \fptype\ tanh(\fptype\ x);

\highlight{constexpr} float tanhf(float x);

\highlight{constexpr} long double tanhl(long double x);

\vspace{2ex}

%% exp %%

\highlight{constexpr} \fptype\ exp(\fptype\ x);

\highlight{constexpr} float expf(float x);

\highlight{constexpr} long double expl(long double x);

\vspace{2ex}

%% exp2 %%

\highlight{constexpr} \fptype\ exp2(\fptype\ x);

\highlight{constexpr} float exp2f(float x);

\highlight{constexpr} long double exp2l(long double x);

\vspace{2ex}

%% expm1 %%

\highlight{constexpr} \fptype\ expm1(\fptype\ x);

\highlight{constexpr} float expm1f(float x);

\highlight{constexpr} long double expm1l(long double x);

\vspace{2ex}

%% frexp %%

\oldhighlight{constexpr} \fptype\ frexp(\fptype\ value, int* exp);

\oldhighlight{constexpr}  float frexpf(float value, int* exp);

\oldhighlight{constexpr}  long double frexpl(long double value, int* exp);

\vspace{2ex}

%% ilogb %%

\oldhighlight{constexpr} \fptype\ ilogb(\fptype\ x);

\oldhighlight{constexpr} int ilogbf(float x);

\oldhighlight{constexpr} int ilogbl(long double x);

\vspace{2ex}

%% ldexp %%

\oldhighlight{constexpr} \fptype\ ldexp(\fptype\ x, int exp);

\oldhighlight{constexpr} float ldexpf(float x, int exp);

\oldhighlight{constexpr} long double ldexpl(long double x, int exp);

\vspace{2ex}

%% log %%

\highlight{constexpr} \fptype\  log(\fptype\ x);

\highlight{constexpr} float logf(float x);

\highlight{constexpr} long double logl(long double x);

\vspace{2ex}

%% log 10 %%

\highlight{constexpr} \fptype\  log10(\fptype\ x);

\highlight{constexpr} float log10f(float x);

\highlight{constexpr} long double log10l(long double x);

\vspace{2ex}

%% loglp %%

\highlight{constexpr} \fptype\ log1p(\fptype\ x);

\highlight{constexpr} float log1pf(float x);

\highlight{constexpr} long double log1pl(long double x);

\vspace{2ex}

%% log2 %%

\highlight{constexpr} \fptype\ log2(\fptype\ x);

\highlight{constexpr} float log2f(float x);

\highlight{constexpr} long double log2l(long double x);

\vspace{2ex}

%% logb %%

\oldhighlight{constexpr} \fptype\ logb(\fptype\ x);

\oldhighlight{constexpr} float logbf(float x);

\oldhighlight{constexpr} long double logbl(long double x);

\vspace{2ex}

%% modf %%

\oldhighlight{constexpr} \fptype\ modf(\fptype\ value, \fptype* iptr);

\oldhighlight{constexpr}  float modff(float value, float* iptr);

\oldhighlight{constexpr}  long double modfl(long double value, long double* iptr);

\vspace{2ex}

%% scalbn %%

\oldhighlight{constexpr} \fptype\ scalbn(\fptype\ x, int n);

\oldhighlight{constexpr} float scalbnf(float x, int n);

\oldhighlight{constexpr} long double scalbnl(long double x, int n);

\vspace{2ex}

%% scalbln %%

\oldhighlight{constexpr} \fptype\ scalbln(\fptype\ x, long int n);

\oldhighlight{constexpr} float scalblnf(float x, long int n);

\oldhighlight{constexpr} long double scalblnl(long double x, long int n);

\vspace{2ex}

\highlight{constexpr} \fptype\ cbrt(\fptype\ x);

\highlight{constexpr} float cbrtf(float x);

\highlight{constexpr} long double cbrtl(long double x);

\vspace{2ex}

//  [c.math.abs], {\it absolute values}

%% abs %%


%% fabs %%

\ldots
\\
%% hypot %%

\highlight{constexpr} \fptype\ hypot(\fptype\ x, \fptype\ y); 

\highlight{constexpr} float hypotf(float x, float y);

\highlight{constexpr} long double hypotl(long double x, long double y);

\vspace{2ex}

// [c.math.hypot3], {\it three-dimensional hypotenuse}

%% hypot3 5%

\highlight{constexpr} \fptype\ hypot(\fptype\ x, \fptype\ y, \fptype\ z);

%% pow %%
\vspace{2ex}

\highlight{constexpr} \fptype\ pow(\fptype\ x, \fptype\ y); 

\highlight{constexpr} float powf(float x, float y);

\highlight{constexpr} long double powl(long double x, long double y);

\vspace{2ex}

%% sqrt %%

\highlight{constexpr} \fptype\ sqrt(\fptype\ x);

\highlight{constexpr} float sqrtf(float x);

\highlight{constexpr} long double sqrtl(long double x);

\vspace{2ex}

%% erf %%

\highlight{constexpr} \fptype\ erf(\fptype\ x);

\highlight{constexpr} float erff(float x);

\highlight{constexpr} long double erfl(long double x);

\vspace{2ex}
%% erfc %%

\highlight{constexpr} \fptype\ erfc(\fptype\ x);

\highlight{constexpr} float erfcf(float x);

\highlight{constexpr} long double erfcl(long double x);

\vspace{2ex}

%% lgamma %%

\highlight{constexpr} \fptype\ lgamma(\fptype\ x);

\highlight{constexpr} float lgammaf(float x);

\highlight{constexpr} long double lgammal(long double x);

\vspace{2ex}
%% tgamma %%

\highlight{constexpr} \fptype\ tgamma(\fptype\ x);

\highlight{constexpr} float tgammaf(float x);

\highlight{constexpr} long double tgammal(long double x);

\ldots

}


\subsection{Modifications to ``Three-dimensional hypotenuse''  [c.math.hypot3]}

\code{
	\highlight{constexpr} \fptype\ hypot(\fptype\ x, \fptype\ y, \fptype\ z);
}

\end{document}
